\section{Theory}
\label{sec:theory}
High-purity germanium (HPGe) detectors are widely used in $\gamma$-ray spectroscopy due
to their excellent energy resolution and well-understood interaction mechanisms.
In this experiment, the detector is employed to calibrate energy, determine the full-energy
detection probability, and analyse characteristic spectral structures of monoenergetic
$\gamma$ radiation. The relevant physical processes are discussed below.


\subsection{Interaction of Photons with Matter and the Extinction Coefficient}
Photons interacting with matter are attenuated according to the exponential law
\begin{equation}
  I(x) = I_0 \, \mathrm{e}^{-\mu(E_\gamma) x},
\end{equation}
where $I_0$ is the incident intensity, $x$ the traversed material thickness, and
$\mu(E_\gamma)$ the energy-dependent extinction coefficient.
The extinction coefficient is the sum of contributions from individual interaction processes,
\begin{equation}
  \mu(E_\gamma) = \mu_\text{ph} + \mu_\text{C} + \mu_\text{pp},
\end{equation}
corresponding to the photoelectric effect, Compton scattering, and pair production.

\begin{figure}[H]
    \centering
    \includegraphics[width=0.5\textwidth]{figures/extinction.png}
    \caption{Energy dependence of the dominant $\gamma$ interaction mechanisms in matter.
    At low energies the photoelectric effect dominates, while Compton scattering prevails
    at intermediate energies. Pair production becomes relevant only above
    \qty{1.022}{\mega\electronvolt} \cite{plot_extinction_germanium}.}
\end{figure}

At low photon energies, the photoelectric effect dominates,
at intermediate energies, Compton scattering is the most probable interaction and depends
primarily on the electron density of the material and
at high energies, pair production becomes increasingly important, but only above the
kinematic threshold of
\begin{equation}
  E_\gamma = 2 m_e c^2 \approx \qty{1.022}{\mega\electronvolt}.
\end{equation}


\subsection{Photoelectric Effect}
In the photoelectric effect, an incident $\gamma$ photon is completely absorbed by a
bound atomic electron, which is ejected from the atom with kinetic energy
\begin{equation}
  E_e = E_\gamma - E_\text{B},
\end{equation}
where $E_\text{B}$ denotes the binding energy of the electron.
The vacancy in the atomic shell is subsequently filled by outer electrons, leading to the
emission of characteristic X-rays or Auger electrons, whose energies are typically absorbed
locally in the detector.

As a result, the full photon energy is deposited within the detector crystal, making the
photoelectric effect the primary mechanism responsible for the formation of the photopeak
in $\gamma$ spectra.
The probability of photoelectric absorption increases strongly with the atomic number $Z$
of the absorber material and decreases rapidly with increasing photon energy.


\subsection{Compton Scattering and Differential Cross Section}
In Compton scattering, a photon transfers part of its energy to an electron and is scattered
under an angle $\theta$. Energy and momentum conservation lead to the Compton formula
\begin{equation}
  E'_\gamma =
  \frac{E_\gamma}{1 + \frac{E_\gamma}{m_e c^2} (1 - \cos\theta)},
  \label{eq:compton}
\end{equation}
where $E'_\gamma$ is the energy of the scattered photon.

The maximum energy transfer occurs for backscattering ($\theta = \pi$), yielding the
Compton edge energy
\begin{equation}
  E_\text{CE} =
  E_\gamma \left(
  1 - \frac{1}{1 + 2 E_\gamma / (m_e c^2)}
  \right).
\end{equation}

The probability of Compton scattering into a given solid angle is described by the
Klein--Nishina differential cross section,
\begin{equation}
  \frac{\mathrm{d}\sigma}{\mathrm{d}\Omega}
  = \frac{r_e^2}{2}
  \left( \frac{E'_\gamma}{E_\gamma} \right)^2
  \left(
    \frac{E'_\gamma}{E_\gamma}
    + \frac{E_\gamma}{E'_\gamma}
    - \sin^2\theta
  \right),
\end{equation}
where $r_e$ is the classical electron radius.
This angular dependence directly shapes the Compton continuum observed in measured spectra.

In many detector applications the scattering angle is not measured directly; instead, the deposited
energy $E$ in the detector is observed. Using the Compton relation
$E'_\gamma = \frac{E_\gamma}{1 + \frac{E_\gamma}{m_e c^2}(1-\cos\theta)}$ and the energy transfer
$E = E_\gamma - E'_\gamma$, the angle $\theta$ can be eliminated in favour of $E$.
The corresponding energy-differential cross section follows from a change of variables,
\begin{equation}
  \frac{\mathrm{d}\sigma}{\mathrm{d}E}
  = \frac{\mathrm{d}\sigma}{\mathrm{d}\Omega}\,
    \frac{\mathrm{d}\Omega}{\mathrm{d}E},
  \qquad
  \mathrm{d}\Omega = 2\pi \sin\theta\,\mathrm{d}\theta,
\end{equation}
which yields the Klein--Nishina distribution written as a function of the electron energy $E$:
\begin{equation}
  \frac{\mathrm{d}\sigma}{\mathrm{d}E}
  = \frac{3}{8}\,\sigma_{\mathrm{Th}}\,
    \frac{1}{m_e c^2}\left[
      2
      + \left(\frac{E}{E_\gamma - E}\right)^2
      \left(
        1 + \frac{E_\gamma - E}{E_\gamma}
        - 2\,\frac{E_\gamma - E}{E}
      \right)
    \right],
  \label{eq:kn_dsig_dE}
\end{equation}
with $\sigma_{\mathrm{Th}}$ the Thomson cross section and $m_e c^2$ the electron rest energy.



\subsection{Pair Production}
At sufficiently high photon energies, $\gamma$ radiation can produce an electron--positron
pair in the Coulomb field of a nucleus. This process becomes possible only if the photon
energy exceeds the combined rest energy of the particles,
\begin{equation}
  E_\gamma \geq 2 m_e c^2 \approx \qty{1.022}{\mega\electronvolt}.
\end{equation}

In pair production, the incident photon is completely absorbed and its energy is converted
into rest mass and kinetic energy of the electron and positron. The probability of this
process increases with photon energy and atomic number $Z$.
After losing its kinetic energy, the positron annihilates with an electron, emitting two
photons of energy \qty{511}{\kilo\electronvolt}.

Depending on whether these annihilation photons are absorbed or escape the detector,
distinct spectral features arise. If one annihilation photon escapes, a \emph{single escape
peak} appears at
\begin{equation}
  E_\text{SE} = E_\gamma - \qty{511}{\kilo\electronvolt},
\end{equation}
while the escape of both photons leads to a \emph{double escape peak} at
\begin{equation}
  E_\text{DE} = E_\gamma - \qty{1022}{\kilo\electronvolt}.
\end{equation}

Although pair production is of minor importance at moderate energies, its characteristic
escape peaks provide a clear experimental signature at higher $\gamma$ energies and
support the interpretation of measured spectra.


\subsection{Signal Formation and Energy Resolution}
In a germanium detector, deposited energy creates electron--hole pairs.
The number of charge carriers is
\begin{equation}
  N = \frac{E_\text{dep}}{\varepsilon},
\end{equation}
with $\varepsilon \approx \qty{2.96}{\electronvolt}$ for germanium.
Here, $\varepsilon$ denotes the mean energy required to produce one electron--hole pair in
germanium. The finite number of charge carriers and their statistical fluctuations impose a fundamental
limit on the achievable energy resolution.

Compared to indirect semiconductors, germanium benefits from a small mean ionisation energy,
leading to narrower intrinsic line widths.
Additional line broadening arises from electronic noise, incomplete charge collection,
and statistical variations in charge amplification.


\subsection{Monochromatic \texorpdfstring{$\gamma$}{gamma} Spectra and Their Interpretation}
A spectrum recorded from a monoenergetic $\gamma$ source exhibits several characteristic
structures:

\begin{figure}[h]
  \centering
  \includegraphics[width=0.9\textwidth]{figures/mono_gamma_spectrum.png}
  \caption{Typical spectral features of a monoenergetic $\gamma$ source.
  Each structure can be traced back to a specific interaction mechanism \cite{physwiki_gamma_spectrum}.}
\end{figure}

\begin{itemize}
  \item \textbf{Photopeak:}  
  Results from complete absorption of the photon energy, typically through a combination
  of Compton scattering followed by a photoelectric interaction. Its narrow width reflects
  the excellent energy resolution of the detector.

  \item \textbf{Compton plateu:}  
  Arises from events where the photon undergoes a single Compton scatter and then escapes
  the detector. The shape of the continuum is governed by the Klein--Nishina cross section.

  \item \textbf{Compton edge:}  
  Marks the maximum energy transfer in a single Compton interaction and corresponds to
  backscattered photons.

  \item \textbf{Backscatter peak:}  
  Caused by photons that are first scattered outside the detector and then fully absorbed
  after re-entering the crystal. Its position is determined by the Compton formula for
  $\theta \approx \pi$.
\end{itemize}

The comparison of measured positions of the Compton edge and backscatter peak with their
theoretical values provides a sensitive consistency check for the energy calibration.

\newpage
\subsection{Full-Energy Detection Probability}
The full-energy detection probability (also referred to as the full-energy peak efficiency)
quantifies the likelihood that an emitted $\gamma$ photon deposits its entire energy in the
detector volume and is therefore registered in the photopeak.
It is defined as
\begin{equation}
  \varepsilon_\text{FE}(E)
  =
  \frac{N_\text{photopeak}(E)}{N_\text{emitted}(E)}.
\end{equation}

In an experiment, the number of photopeak events is obtained from the background-corrected
line content of the corresponding full-energy peak.
The number of emitted photons of energy $E$ is given by
\begin{equation}
  N_\text{emitted}(E)
  =
  A \cdot t_\text{live} \cdot I_\gamma(E),
\end{equation}
where $A$ denotes the activity of the source, $t_\text{live}$ the live time of the
measurement, and $I_\gamma(E)$ the emission probability of the $\gamma$ transition.

Combining these expressions yields the experimentally relevant form of the full-energy
detection probability:
\begin{equation}
  \boxed{
  \varepsilon_\text{FE}(E)
  =
  \frac{N_\text{photopeak}(E)}
  {A \cdot t_\text{live} \cdot I_\gamma(E)}
  }.
\end{equation}

The full-energy detection probability depends on the detector material and dimensions, the
source--detector geometry, and the photon energy.
At low photon energies, the photoelectric effect dominates, resulting in a high probability
of complete energy absorption.
With increasing energy, Compton scattering becomes more likely, and photons may escape the
detector after partial energy deposition.
At even higher energies, processes such as pair production and the escape of annihilation
photons further reduce the contribution to the full-energy peak.

Consequently, despite a non-negligible total interaction probability, only a fraction of
detected events contributes to the photopeak, leading to a decreasing
$\varepsilon_\text{FE}$ with increasing photon energy.


\subsection{Solid Angle of a Circular Detector}
The solid angle $\Omega$ subtended by a detector surface $A$ at the position of a point
source is defined as
\begin{equation}
  \Omega = \int_A \frac{\cos\theta}{r^2}\,\mathrm{d}A,
\end{equation}
where $r$ denotes the distance from the source to the surface element $\mathrm{d}A$ and
$\theta$ is the angle between the surface normal and the line of sight.

For a flat circular detector of radius $R$ located at a distance $d$ from the source and
oriented perpendicular to the source--detector axis, cylindrical coordinates
$(\rho,\varphi)$ may be used.
In this geometry, $r = \sqrt{d^2 + \rho^2}$ and $\cos\theta = d / \sqrt{d^2 + \rho^2}$, while
the surface element is given by $\mathrm{d}A = \rho\,\mathrm{d}\rho\,\mathrm{d}\varphi$.
Inserting these expressions yields
\begin{equation}
  \Omega
  =
  \int_0^{2\pi}\!\!\int_0^{R}
  \frac{d\,\rho}{(d^2 + \rho^2)^{3/2}}
  \,\mathrm{d}\rho\,\mathrm{d}\varphi.
\end{equation}

Evaluating the integrals leads to the closed-form expression
\begin{equation}
  \Omega
  =
  2\pi
  \left(
  1 - \frac{d}{\sqrt{d^2 + R^2}}
  \right),
\end{equation}
which describes the solid angle subtended by a circular detector as a function of its
radius and the source--detector distance.
The solid angle determines the purely geometrical fraction of isotropically emitted
$\gamma$ quanta that reach the detector.
Normalising the solid angle to the full solid angle $4\pi$ yields the geometrical
acceptance $\Omega / 4\pi$.

The total detection probability is then obtained by combining this geometrical factor
with the full-energy detection probability $\varepsilon_\text{FE}(E)$, which accounts for
the interaction and absorption processes within the detector material:
\begin{equation}
  P_\text{det}(E)
  =
  \frac{4\pi}{\Omega}
  \cdot
  \varepsilon_\text{FE}(E).
\end{equation}
This expression shows that the probability for detecting a $\gamma$ quantum in the
full-energy peak is determined by both the source--detector geometry and the intrinsic
detector response.

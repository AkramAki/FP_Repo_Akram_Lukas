\section{Experimental Setup and Procedure}
\label{sec:procedure}
The setup consists of a cylindrical germanium detector, whith a mount for radioactive sources on top. 
It is enclosed in a lead shield, and cooled with liquid nitrogen from below. 
The detector is supplied with a voltage of $\SI{5}{\volt}$ and connected to a multichannel analyzer. 
The detector has a Lithium n-doped surface with a thickness of $\SI{0.5}{\centi\meter}$ and a gold layer on the inside, 
which acts as a p strong doping, with a thickness of $\SI{20}{\micro\meter}$. 
The detector is aluminum coated on the outside to ensure protection. 
The setup is depicted in \autoref{fig:setup}. \\
\begin{figure}[h]
    \centering
    \includegraphics[width=0.9\textwidth]{v18_germanium/figures/setup.png}
    \caption{Schematic of the germanium detector setup \cite{v18}.}
    \label{fig:setup}
\end{figure}
During the experiment multiple sources are placed on the mount above the detector 
and their spectra are measured via a computer which is connected to the multichannel analyzer. 
The following spectra are recorded:
\begin{itemize}
    \item Europium-152
    \item Cesium-137
    \item Barium-133
    \item One unknown source
    \item Background radiation
\end{itemize}
Each spectrum with a source is recorded for a duration of about $\SI{90}{\minute}$, and the background radiation over one night. 

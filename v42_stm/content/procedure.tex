\section{Experimental Setup and Procedure}
\label{sec:procedure}
In the following, the experimental procedure for operating the STM 
and acquiring data on HOPG and gold is described. 
The procedure follows the instructions provided for experiment V42.


\subsection{Preparation and cleaning}
Before starting the experiment, all participants wore protective gloves to avoid contamination. 
All tools used during the experiment, including the STM setup, probe holder, and the wire used for 
tip fabrication, were thoroughly cleaned using ethanol and lint-free wipes that do not scratch surfaces 
or leave dust or cloth residues. This step is crucial, as even small contaminations can significantly
affect the tunneling current and image quality.


\subsection{Tip fabrication}
The STM tip was fabricated from a cleaned metal wire of "unknown" material. A short section of wire was cut 
and held firmly at one end using tweezers. A wire cutter was then applied at an angle of approximately $\qty{45}{\deg}$ 
at the position where the tip was to be formed. The cutter was pressed lightly into the wire to create a small 
notch and then pulled to tear off a piece of the wire. This method aims to produce a very sharp tip, 
ideally terminating in a single atom.


\subsection{Sample preparation and mounting}
The HOPG sample was initially used as provided and assumed to be pre-prepared. In later stages of the 
experiment, the HOPG surface was cleaved using tape to test whether surface contamination rather than tip quality 
limited the measurements. The gold sample was used as provided.

The STM tip was mounted first and positioned below a small gold lever. After mounting, the probe assembly was inserted 
into the STM. Care was taken not to exert mechanical force on the gold lever during insertion, as this 
could damage the setup.


\subsection{Approach procedure}
The sample was moved towards the tip using the manual coarse approach controlled by the STM software. During 
this process, the reflection of the tip on the sample surface was observed. When the tip and its reflection 
appeared very close, the manual approach was stopped to prevent mechanical contact.

Subsequently, the automatic approach function of the STM control software was activated. The software then adjusted 
the tip--sample distance until a tunneling current was detected. A successful approach was indicated by a stable signal 
and a green status indicator in the software interface.


\subsection{Measurement procedure on HOPG}
For HOPG measurements, the initial scan size was set to $200\,\mathrm{nm}$ (or full screen). The tunneling current was 
set to $1\,\mathrm{A}$ and the scan speed to $0.2\,\mathrm{lines/s}$. Under optimal conditions, a flat scan line should 
be visible. A straight but tilted line indicates that the sample is slightly angled, while a fuzzy or unstable line 
suggests insufficient tip quality.

If a stable and sufficiently flat scan was obtained, the scan size was gradually reduced from $200\,\mathrm{nm}$ to 
$40$--$30\,\mathrm{nm}$ and finally to $2$--$2.5\,\mathrm{nm}$. At small scan sizes, the scan speed was reduced to 
$0.06\,\mathrm{lines/s}$ to improve image stability. At the smallest scan sizes, the atomic lattice structure of HOPG
should become visible.

In practice, obtaining a sufficiently flat and stable scan proved difficult. Although a mostly flat scan was 
achieved once, the STM tip was later damaged, and repeated attempts to fabricate a suitable tip were largely unsuccessful.


\subsection{Measurement procedure on gold}
Gold measurements were intended to be performed using a tip that had already produced atomic resolution images on 
HOPG. The approach procedure was identical to that used for HOPG. For gold, the tunneling current was set to 
$1.5\,\mathrm{A}$, the scan size to full screen, and the scan speed to $0.3\,\mathrm{lines/s}$.

During the experiment, the STM tip was damaged by moving too far into the sample, making further measurements with that tip impossible.


\subsection{Data acquisition and storage}
After each scan, the STM control software automatically saved the acquired data. The data were stored in the native 
\texttt{.nid} file format of the software. After the experiment, suitable measurement files were identified and 
extracted for further analysis.

Due to the limited amount of usable data obtained during the experiment, additional STM data sets in \texttt{.nid} 
format were provided by the experiment supervisor from other student groups. These data were used to complete the 
analysis tasks. The reason behind the limited amount of data stems from the fact that most of the produced tips where 
not sufficient enough to create useable data and thus our experimental time was used up on learning how to create tips.

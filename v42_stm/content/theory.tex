\section{Theory}
\label{sec:theory}
In this chapter the theoretical background that is needed to describe the scanning tunneling microscope (stm) is laid down. 
\subsection{Tunneling}
The quantum mechanical phenomenon of tunneling arises, because the wave function $\psi$ that is obtained by solving the 
Schrödinger equation for a given potential $V(x)$ has nonzero values at positions $x$ which, classically, would be considered a 
potential barrier. 
Therefore particles described by this wavefunction also have a nonzero probability to be measured in or behind a potential barrier. 
If those particles are charged one can measure the tunneling as a current. 
This tunneling current is approximated as
\begin{equation*}
    . %% Gleichung einfügen
\end{equation*}
\subsection{Piezo crystals}
Piezo crystals are defined by their property to change their polarization when mechanical stress is applied and inversely 
expand or shrink when voltage is applied, which is called the inverse piezoelectric effect. 
The relation between expansion and voltage is ideally linear. 
Piezo crystals are built by heating them up, 
then polarizing them and cooling down again in order to ensure a long living polarization. 

\subsection{Layout of a scanning tunneling microscope}
A scanning tunneling microscope consists of a thin and conducting tip, which receives tunneling current. 
In order to obtain a one atom resolution, the end of the tip needs to be only one atom wide. 
The potential barrier is realized by the medium in between the tip and sample. 
Moreover the microscope is made of piezo tubes which move the sample via the inverse piezoelectric effect, which allows 
piezo crystals to expand linear to a voltage that is applied. 
By using multiple tubes and different voltages which drive the tubes in different directions a controlled movement in x-,y- and z- 
direction is possible. 
The sample can then be moved in z-direction until a tunneling current is measurable and scanned along x- and y-direction to analyze 
the lattice structure of the sample. 

\subsection{Errors and corrections}
Multiple errors can occur while scanning the sample. 
Some of them arise from inaccuracies of the piezos, such as nonlinearity, hysteresis, creep, cross coupling and aging. 
Nonlinearity between voltage and expansion of the piezos causes uncertainties in sample position, 
while hysteresis causes a difference in results when changing the propagation direction of the scan. 
Creep describes a slow drift of piezo expansion after a voltage spike which also disturbs the movement accuracy. 
Cross coupling means that x-,y- and z-movement of the piezos are not completely decoupled 
and therefore the movement along the sample might not be straight but distorted. 
Aging is important when piezos have not been used for longer periods of time, 
as the polarization is lost and the response to voltage inputs is smaller. 
The other way around the response is larger when they have been used frequently. 
On top of that the measurements might be disturbed by vibrations and thermal expansion 
as the stm is sensitive to small changes in distance. 
Most of these errors have to be corrected mathematically, 
while hysteresis can be compensated by seperating measurements in different directions. 

\subsection{PID-controlling}
A PID-controller is used to control an oscillating value $V$ to a certain setpoint $S$ by regulating an input $I$.
To realize this three terms are used which which contribute to the input as following:
\begin{equation*}
    I(V)=P\cdot(S-V)+I\cdot\int{(S-V)}\;\mathrm{d}t+D\cdot\frac{\mathrm{d}(S-V)}{\mathrm{d}t}\,,
\end{equation*}
where $S-V$ is the error of the current value compared to the setpoint. 

\subsection{Highly oriented pyrolytic graphite}
Highly oriented pyrolytic graphite (HOPG) is a extremely periodic form of graphite, where misorientation is small. 
It has a hexagonal crystal structure which is layered in 2d-layers that are bound by van-der-Waals interaction. 
Since two layers are always misplaced compared to the neighbouring layers, 
there are two nonequivalent positions in the hexagonal structure:
Some Atoms have an atom right below it, while others do not. 
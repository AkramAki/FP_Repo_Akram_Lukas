\section{Analysis}% Introduce software
\label{sec:analysis}

\subsection{Gold}
The forward measurement of the Gold sample is displayed in \autoref{fig:gold_forward}, 
where height information is displayed along the $x$-$y$-plane. 
\begin{figure}
   \centering
   \includegraphics[width =0.7\textwidth]{analysis/gold/profiles_forward.png}
   \caption{Forward measurement of the gold sample with profile slices.}
   \label{fig:gold_forward}
\end{figure}
The height of the tip was changed according to the method of constant current, see \autoref{sec:theory}. 
This dataset was available from previous measurements. 
It is not possible to resolve single atoms in a gold sample, as gold is a conductor, therefore the electrons which create the tip current are strongly delocalized 
and do not depict a crystal structure. 
In \autoref{fig:gold_forward} multiple lines are marked, which indicate profiles that where taken to obtain the height of jumps in the data. 
The backwards measurement is displayed in \autoref{fig:gold_backwards}, but should to be adjusted, as the white stripes on the right side of 
the picture indicate that the tip was dislocated from the sample and therefore skew the color scheme. 
\begin{figure}
   \centering
   \includegraphics[width =0.7\textwidth]{analysis/gold/backwards_scan_raw.png}
   \caption{Raw backwards measurement of the gold sample.}
   \label{fig:gold_backwards}
\end{figure}
The color scale is manually adjusted so that the higher end corresponds to the value of the forward scan. 
This corrected picture in \autoref{fig:gold_backwards_corr} is used to read out more profiles with jumps and obtain their height analogously 
to the forward scan. 
\begin{figure}
   \centering
   \includegraphics[width =0.7\textwidth]{analysis/gold/profiles_backwards_corr.png}
   \caption{Corrected backwards measurement of the gold sample with profile slices.}
   \label{fig:gold_backwards_corr}
\end{figure}
Each forward profile is pictured in \autoref{fig:gold_profiles_forwards} together with the readout of jump height provided from the software, 
where vertical lines mark the beginning and end of each jump. 
Analogously backwards profiles are pictured in \autoref{fig:gold_profiles_backwards}.
\begin{figure}
    \centering
    \begin{subfigure}{0.9\textwidth}
        \centering
        \includegraphics[width =1\textwidth]{analysis/gold/forward_line_1.png}
    \end{subfigure}
    \begin{subfigure}{0.9\textwidth}
        \centering
        \includegraphics[width =1\textwidth]{analysis/gold/forward_line_2.png}
    \end{subfigure}
    \begin{subfigure}{0.9\textwidth}
        \centering
        \includegraphics[width =1\textwidth]{analysis/gold/forward_line_3.png}
    \end{subfigure}
    \caption{Profiles obtained from forwards scan slices with jump height readout.}
    \label{fig:gold_profiles_forwards}
\end{figure}
\begin{figure}
    \centering
    \begin{subfigure}{0.8\textwidth}
        \centering
        \includegraphics[width =1\textwidth]{analysis/gold/backwards_line_1.png}
    \end{subfigure}
    \begin{subfigure}{0.8\textwidth}
        \centering
        \includegraphics[width =1\textwidth]{analysis/gold/backwards_line_2.png}
    \end{subfigure}
    \begin{subfigure}{0.8\textwidth}
        \centering
        \includegraphics[width =1\textwidth]{analysis/gold/backwards_line_3.png}
    \end{subfigure}
    \caption{Profiles obtained from backwards scan slices with jump height readout.}
    \label{fig:gold_profiles_backwards}
\end{figure}
The resulting heights are listed in \autoref{tab:heights}. 
The average height is \begin{equation*}
    \bar{d} = \SI{6.79 \pm 2.52}{\nano\meter}\,,
\end{equation*}
where the uncertainty is the standard deviation of the measured heights. 
\begin{table}
   \centering
   \caption{Jump heights in gold scans.}
   \label{tab:heights}
   \sisetup{table-format = 3.0} 
   \begin{tblr}{
       colspec = {S S},
       row{1} = {guard, mode = math}
   }
    \toprule 
        \text{Profile} & \lvert d \rvert \mathbin{/} \unit{\nano\meter} \\
    \midrule
        \text{Forwards scan 1} & 4.14 \\
        \text{Forwards scan 2} & 4.86 \\
        \text{Forwards scan 3} & 8.25 \\
        \text{Backwards scan 1} & 5.21 \\
        \text{Backwards scan 2} & 6.75 \\
        \text{Backwards scan 3} & 11.55 \\
    \bottomrule
   \end{tblr}
\end{table}

% templates

%\begin{table}
%    \centering
%    \caption{.}
%    \label{tab:}
%    \sisetup{table-format = 3.0} 
%    \begin{tblr}{
%        colspec = {S S S},
%        row{1} = {guard, mode = math}
%    }
%        \toprule 
%        x \mathbin{/} \unit{\milli\meter} \\
%        \midrule
%        %data here
%        \bottomrule
%    \end{tblr}
%\end{table}

%\begin{figure}
%    \centering
%    \includegraphics[width =0.48\textwidth]{}
%    \caption{.}
%    \label{fig:}
%\end{figure}

%\begin{wrapfigure}[20]{r}{0.5\textwidth}
%    \begin{center}
%        \includegraphics[width =0.48\textwidth]{figures/}
%        \caption{}
%        \label{fig:}
%    \end{center}
%\end{wrapfigure}
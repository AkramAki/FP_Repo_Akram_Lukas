\section{Analysis}
\label{sec:analysis}

\subsection{Gold}
The forward measurement of the Gold sample is displayed in \autoref{fig:gold_forward}, 
where height information is displayed along the x-y-plane. 
The height of the tip was changed according to the method of constant current, see \autoref{chap:theory}. %% Referenzen double checken
This dataset was available from previous measurements. 
In \autoref{fig:gold_forward} multiple lines are marked, which indicate profiles that where taken to obtain the height of jumps in the data. 
The backwards measurement is displayed in \autoref{fig:gold_backwards}, but has to be corrected, as the white stripes on the right side of 
the picture indicate that the tip was dislocated from the sample. 
Therefore the faulty part auf the data was cut out and the colour scheme was rescaled for the remainder of the measurement. 
This corrected picture in \autoref{fig:gold_backwards_corr} is used to read out more profiles with jumps and obtain their height analogously 
to the forward scan. 
Each profile is pictured in \autoref{fig:gold_profiles}
The resulting heights are listed in \autoref{tab:heights}
% templates

%\begin{table}
%    \centering
%    \caption{.}
%    \label{tab:}
%    \sisetup{table-format = 3.0} 
%    \begin{tblr}{
%        colspec = {S S S},
%        row{1} = {guard, mode = math}
%    }
%        \toprule 
%        x \mathbin{/} \unit{\milli\meter} \\
%        \midrule
%        %data here
%        \bottomrule
%    \end{tblr}
%\end{table}

%\begin{figure}
%    \centering
%    \includegraphics[width =0.48\textwidth]{}
%    \caption{.}
%    \label{fig:}
%\end{figure}

%\begin{wrapfigure}[20]{r}{0.5\textwidth}
%    \begin{center}
%        \includegraphics[width =0.48\textwidth]{figures/}
%        \caption{}
%        \label{fig:}
%    \end{center}
%\end{wrapfigure}
\section{Analysis}% Introduce software
\label{sec:analysis}
For the analysis of the STM measurements, the software Gwyddion \cite{gwyddion} is used.
\subsection{HOPG}
There are multiple HOPG measurements where the method of constant current was used, 
which are depicted in \autoref{fig:hopg_1} and \autoref{fig:hopg_2} and were taken on different days. 
\begin{figure}
    \centering
    \begin{subfigure}{0.48\textwidth}
        \centering
        \includegraphics[width =1\textwidth]{analysis/HOPG/Scan_1_forwards.png}
    \end{subfigure}
    \begin{subfigure}{0.48\textwidth}
        \centering
        \includegraphics[width =1\textwidth]{analysis/HOPG/Scan_1_backwards.png}
    \end{subfigure}
    \begin{subfigure}{0.48\textwidth}
        \centering
        \includegraphics[width =1\textwidth]{analysis/HOPG/Scan_2_forwards.png}
    \end{subfigure}
        \begin{subfigure}{0.48\textwidth}
        \centering
        \includegraphics[width =1\textwidth]{analysis/HOPG/Scan_2_backwards.png}
    \end{subfigure}
        \begin{subfigure}{0.48\textwidth}
        \centering
        \includegraphics[width =1\textwidth]{analysis/HOPG/Scan_3_forwards.png}
    \end{subfigure}
        \begin{subfigure}{0.48\textwidth}
        \centering
        \includegraphics[width =1\textwidth]{analysis/HOPG/Scan_3_backwards.png}
    \end{subfigure}
    \caption{HOPG stm scans with lattice vectors over multiple unit cells.}
    \label{fig:hopg_1}
\end{figure}
\begin{figure}
    \centering
    \begin{subfigure}{0.48\textwidth}
        \centering
        \includegraphics[width =1\textwidth]{analysis/HOPG/Scan_4_forwards.png}
    \end{subfigure}
        \begin{subfigure}{0.48\textwidth}
        \centering
        \includegraphics[width =1\textwidth]{analysis/HOPG/Scan_4_backwards.png}
    \end{subfigure} 
    \begin{subfigure}{0.48\textwidth}
        \centering
        \includegraphics[width =1\textwidth]{analysis/HOPG/Scan_5_forwards.png} 
    \end{subfigure}
        \begin{subfigure}{0.48\textwidth}
        \centering
        \includegraphics[width =1\textwidth]{analysis/HOPG/Scan_5_backwards.png}
    \end{subfigure} 
    \caption{HOPG stm scans with lattice vectors over multiple unit cells.}
    \label{fig:hopg_2}
\end{figure}

The pictures show the leveled data, where remaining tilt in the signal was removed by the software. 
The lattices vectors are added manually into the picture and scaled over multiple unit cells to reduce uncertainty. 
Length of those vectors are provided by the software and listet in \autoref{tab:hopg_distances}. 
\begin{table}
   \centering
   \caption{Lattice vector lengths read out from real space.}
   \label{tab:hopg_distances}
   \sisetup{table-format = 1.3} 
   \begin{tblr}{
       colspec = {S S S},
       row{1} = {guard, mode = math}
   }
       \toprule 
       d \mathbin{/} \unit{\nano\meter} \\
       \midrule
       0.222 \\
       0.202 \\
       0.195 \\
       0.215 \\
       0.200 \\
       0.173 \\
       0.210 \\
       0.218 \\
       0.189 \\
       0.178 \\
       0.188 \\
       0.184 \\
       0.161 \\
       0.195 \\
       0.200 \\
       0.188 \\
       0.183 \\
       0.181 \\
       0.213 \\
       0.180 \\
       0.179 \\
       0.214 \\
       0.182 \\
       0.180 \\
       \bottomrule
   \end{tblr}
\end{table}
Each value is estimated to have an uncertainty of $\pm\SI{0.15}{\nano\meter}$ 
due to manual readout and disambiguities when identifiying atoms in the image.
The resulting mean distance is 
\begin{equation*}
    \bar{d} = \SI{0.188\pm0.026}{\nano\meter}\,.
\end{equation*} 
One of the measurements was not used for this analysis, as the atoms could not be properly identified. 
Furthermore the two dimensional fourier transform of each measurement 
was calculated by the software and is depicted in \autoref{fig:hopg_fft_1} and \autoref{fig:hopg_fft_2}. 
\begin{figure}
    \centering
    \begin{subfigure}{0.43\textwidth}
        \centering
        \includegraphics[width =1\textwidth]{analysis/HOPG/FFT/Scan_1_FFT_1.png}
    \end{subfigure}
    \begin{subfigure}{0.55\textwidth}
        \centering
        \includegraphics[width =1\textwidth]{analysis/HOPG/FFT/Scan_1_FFT_2.png}
    \end{subfigure}
    \begin{subfigure}{0.48\textwidth}
        \centering
        \includegraphics[width =1\textwidth]{analysis/HOPG/FFT/Scan_2_FFT_1.png}
    \end{subfigure}
        \begin{subfigure}{0.48\textwidth}
        \centering
        \includegraphics[width =1\textwidth]{analysis/HOPG/FFT/Scan_2_FFT_2.png}
    \end{subfigure}
        \begin{subfigure}{0.48\textwidth}
        \centering
        \includegraphics[width =1\textwidth]{analysis/HOPG/FFT/Scan_3_FFT_1.png}
    \end{subfigure}
        \begin{subfigure}{0.48\textwidth}
        \centering
        \includegraphics[width =1\textwidth]{analysis/HOPG/FFT/Scan_3_FFT_2.png}
    \end{subfigure}
    \caption{HOPG two dimensional fourier transforms of the stm scans with reciprocal lattice vectors.}
    \label{fig:hopg_fft_1}
\end{figure}
\begin{figure}
    \centering
    \begin{subfigure}{0.43\textwidth}
        \centering
        \includegraphics[width =1\textwidth]{analysis/HOPG/FFT/Scan_4_FFT_1.png}
    \end{subfigure}
        \begin{subfigure}{0.55\textwidth}
        \centering
        \includegraphics[width =1\textwidth]{analysis/HOPG/FFT/Scan_4_FFT_2.png}
    \end{subfigure} 
    \begin{subfigure}{0.48\textwidth}
        \centering
        \includegraphics[width =1\textwidth]{analysis/HOPG/FFT/Scan_5_FFT_1.png} 
    \end{subfigure}
        \begin{subfigure}{0.48\textwidth}
        \centering
        \includegraphics[width =1\textwidth]{analysis/HOPG/FFT/Scan_5_FFT_2.png}
    \end{subfigure} 
    \caption{HOPG two dimensional fourier transforms of the stm scans with reciprocal lattice vectors.}
    \label{fig:hopg_fft_2}
\end{figure}
There the reciprocal lattice is visibile, and its lattice vectors were also included manually. 
Analogous to before the lengths were read out and since the length of the reciprocal lattice vectors is inverse
to the real space lattice vectors, the real space lattice vectors where calculated from this data. 
The resulting distances are listed in \autoref{tab:hopg_distances_fft}.
The one measurement not used before was also not used here for consistency.
\begin{table}
   \centering
   \caption{Lattice vector lengths resulting from the two dimensional fourier transform.}
   \label{tab:hopg_distances_fft}
   \sisetup{table-format = 1.3} 
   \begin{tblr}{
       colspec = {S},
       row{1} = {guard, mode = math}
   }
       \toprule 
       d_{\text{FFT}} \mathbin{/} \unit{\nano\meter} \\
       \midrule
        0.161 \\ 
        0.165 \\ 
        0.164 \\
        0.167 \\ 
        0.120 \\ 
        0.164 \\
        0.141 \\ 
        0.135 \\ 
        0.152 \\ 
        0.134 \\ 
        0.164 \\ 
        0.132 \\
        0.132 \\ 
        0.164 \\ 
        0.167 \\ 
        0.132 \\
       \bottomrule
   \end{tblr}
\end{table}
And the resulting mean distance is
\begin{equation*}
    \bar{d}_\text{FFT} = \SI{0.150\pm0.007}{\nano\meter}\,.
\end{equation*}
Analogous to before the uncertainty is estimated from manual readout and identification of the peaks in the image and accounted for with an uncertainty of $\pm\SI{1.2}{\nano\meter^{-1}}$ in reciprocal space which results in 
$\Delta d_\text{FFT} \approx \pm\SI{0.03}{\nano\meter}$ in real space after inversion. 
\subsection{Gold}
The forward measurement of the Gold sample is displayed in \autoref{fig:gold_forward}, 
where height information is displayed along the $x$-$y$-plane. 
\begin{figure}
   \centering
   \includegraphics[width =0.7\textwidth]{analysis/gold/profiles_forward.png}
   \caption{Forward measurement of the gold sample with profile slices.}
   \label{fig:gold_forward}
\end{figure}
The height of the tip was changed according to the method of constant current, see \autoref{sec:theory}. 
This dataset was available from previous measurements. 
It is not possible to resolve single atoms in a gold sample, as gold is a conductor, therefore the electrons which create the tip current are strongly delocalized 
and do not depict a crystal structure. 
In \autoref{fig:gold_forward} multiple lines are marked, which indicate profiles that where taken to obtain the height of jumps in the data. 
The backwards measurement is displayed in \autoref{fig:gold_backwards}, but should to be adjusted, as the white stripes on the right side of 
the picture indicate that the tip was dislocated from the sample and therefore skew the color scheme. 
\begin{figure}
   \centering
   \includegraphics[width =0.7\textwidth]{analysis/gold/backwards_scan_raw.png}
   \caption{Raw backwards measurement of the gold sample.}
   \label{fig:gold_backwards}
\end{figure}
The color scale is manually adjusted so that the higher end corresponds to the value of the forward scan. 
This corrected picture in \autoref{fig:gold_backwards_corr} is used to read out more profiles with jumps and obtain their height analogously 
to the forward scan. 
\begin{figure}
   \centering
   \includegraphics[width =0.7\textwidth]{analysis/gold/profiles_backwards_corr.png}
   \caption{Corrected backwards measurement of the gold sample with profile slices.}
   \label{fig:gold_backwards_corr}
\end{figure}
Each forward profile is pictured in \autoref{fig:gold_profiles_forwards} together with the readout of jump height provided from the software, 
where vertical lines mark the beginning and end of each jump. 
Analogously backwards profiles are pictured in \autoref{fig:gold_profiles_backwards}.
\begin{figure}
    \centering
    \begin{subfigure}{0.6\textwidth}
        \centering
        \includegraphics[width =1\textwidth]{analysis/gold/forward_line_1.png}
    \end{subfigure}
    \begin{subfigure}{0.6\textwidth}
        \centering
        \includegraphics[width =1\textwidth]{analysis/gold/forward_line_2.png}
    \end{subfigure}
    \begin{subfigure}{0.6\textwidth}
        \centering
        \includegraphics[width =1\textwidth]{analysis/gold/forward_line_3.png}
    \end{subfigure}
    \caption{Profiles obtained from forwards scan slices with jump height readout.}
    \label{fig:gold_profiles_forwards}
\end{figure}
\begin{figure}
    \centering
    \begin{subfigure}{0.6\textwidth}
        \centering
        \includegraphics[width =1\textwidth]{analysis/gold/backwards_line_1.png}
    \end{subfigure}
    \begin{subfigure}{0.6\textwidth}
        \centering
        \includegraphics[width =1\textwidth]{analysis/gold/backwards_line_2.png}
    \end{subfigure}
    \begin{subfigure}{0.6\textwidth}
        \centering
        \includegraphics[width =1\textwidth]{analysis/gold/backwards_line_3.png}
    \end{subfigure}
    \caption{Profiles obtained from backwards scan slices with jump height readout.}
    \label{fig:gold_profiles_backwards}
\end{figure}
The resulting heights are listed in \autoref{tab:heights}. 
The average height is \begin{equation*}
    \bar{d} = \SI{6.79 \pm 2.52}{\nano\meter}\,,
\end{equation*}
where the uncertainty is the standard deviation of the measured heights. 
\begin{table}
   \centering
   \caption{Jump heights in gold scans.}
   \label{tab:heights}
   \sisetup{table-format = 3.0} 
   \begin{tblr}{
       colspec = {S S},
       row{1} = {guard, mode = math}
   }
    \toprule 
        \text{Profile} & \lvert d \rvert \mathbin{/} \unit{\nano\meter} \\
    \midrule
        \text{Forwards scan 1} & 4.14 \\
        \text{Forwards scan 2} & 4.86 \\
        \text{Forwards scan 3} & 8.25 \\
        \text{Backwards scan 1} & 5.21 \\
        \text{Backwards scan 2} & 6.75 \\
        \text{Backwards scan 3} & 11.55 \\
    \bottomrule
   \end{tblr}
\end{table}

% templates

%\begin{table}
%    \centering
%    \caption{.}
%    \label{tab:}
%    \sisetup{table-format = 3.0} 
%    \begin{tblr}{
%        colspec = {S S S},
%        row{1} = {guard, mode = math}
%    }
%        \toprule 
%        x \mathbin{/} \unit{\milli\meter} \\
%        \midrule
%        %data here
%        \bottomrule
%    \end{tblr}
%\end{table}

%\begin{figure}
%    \centering
%    \includegraphics[width =0.48\textwidth]{}
%    \caption{.}
%    \label{fig:}
%\end{figure}

%\begin{wrapfigure}[20]{r}{0.5\textwidth}
%    \begin{center}
%        \includegraphics[width =0.48\textwidth]{figures/}
%        \caption{}
%        \label{fig:}
%    \end{center}
%\end{wrapfigure}
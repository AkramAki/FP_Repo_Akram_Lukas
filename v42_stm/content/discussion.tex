\section{Discussion}
\label{sec:discussion}
\subsection{HOPG}
The analysis of the HOPG sample in \autoref{sec:analysis} results in an average lattice constant of 
\begin{equation*}
    \bar{d} = \SI{0.188\pm0.026}{\nano\meter}\,,
\end{equation*}
which is smaller than the literature value of \SI{0.246}{\nano\meter} \cite{Güntherodt}. 
Also the value obtained from the fourier transform analysis is smaller than the literature value,
\begin{equation*}
    \bar{d}_\text{FFT} = \SI{0.150\pm0.007}{\nano\meter}\,.
\end{equation*}
The deviations could be due to misorientation of the sample, 
such that the the image plain does not align with the 2D honeycomb lattice. 
The analysis by fourier transform is much more precise, 
which is due to blurry single atom pictures in the real space images, which are reduced to more sharp peaks in fourier space. 

\subsection{Gold}
The analysis of the gold sample in \autoref{sec:analysis} results in an avergage height of
\begin{equation*}
    \bar{d} = \SI{6.79 \pm 2.52}{\nano\meter}\,,
\end{equation*}
which is much larger than the lattice constant of gold of \SI{0.408}{\nano\meter} \cite{gold_lattice}. 
Most likely the jumps that were analyzed are larger than a single atom layer distance. 
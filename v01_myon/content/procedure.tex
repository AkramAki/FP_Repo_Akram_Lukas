\newpage
\section{Experimental Setup and Procedure}
\label{sec:procedure}

The measurement of the muon lifetime was carried out using a scintillation detector tank equipped with two photomultiplier 
tubes (PMTs) at each end and a time-resolved electronic readout chain. Prior to data acquisition, all electronic components 
were powered on and allowed to stabilize.

First, the high-voltage supply for the PMTs was activated and adjusted such that the PMTs produce varying pulse shapes and amplitudes which is inspectable using a prior connected oscilloscope.

Each PMT signal was then routed through a discriminator, where the threshold level was set such that noise pulses were
suppressed while genuine scintillation signals were reliably detected which was the case when approx. 30 pulses per second are detected. 
The discriminator settings were also tuned to achieve comparable count rates on both channels.

To ensure simultaneous detection of scintillation light at both ends of the detector, adjustable delay lines were used to compensate for differences in signal propagation times. The relative delay was optimized by monitoring the output rate of the coincidence unit while varying one delay setting. The delay corresponding to the maximum coincidence rate was selected for the measurement.

After establishing a stable coincidence signal, the timing logic was completed. A coincidence pulse generated by an incoming muon initiated a timing window during which a possible decay signal 
could be registered. The duration of this search window was defined by a monostable circuit and chosen to be significantly longer than the expected muon lifetime. This was done by feeding 
the coincidence output into two AND gates, another delay line with $\Delta t = \qty{30}{\nano\second}$ and a monoflop. If a second coincidence occurred within this window, it was interpreted as a decay event.

The time interval between the initial muon signal and the subsequent decay signal was converted into an analog voltage using a time-to-amplitude converter (TAC). This voltage was digitized and recorded by a multichannel analyzer (MCA), which accumulated a histogram of measured time intervals.

Prior to the actual muon measurement, the MCA was calibrated using a pulse generator that produced pairs of electronic pulses with known time separations. The resulting channel positions were used to establish a linear relationship between MCA channel number and physical time.

Following calibration, the detector was operated continuously over an extended measurement period to collect a statistically significant number of decay events. The resulting time distribution served as the basis for determining the mean lifetime of cosmic muons.

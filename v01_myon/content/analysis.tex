\section{Analysis}
\label{sec:analysis}
To calibrate the MCA the double pulse delay was varied, which causes a peak to appear at the channel corresponding to the delay time. 
\autoref{tab:calibration} shows the measured peak positions for different delay times. 
\begin{table}
   \centering
   \caption{Measured peak positions for different delay times $t$.}
   \label{tab:calibration}
   \sisetup{table-format = 3.0} 
   \begin{tblr}{
       colspec = {S[table-format=1.1] S},
       row{1} = {guard, mode = math}
   }
       \toprule 
       t \mathbin{/} \unit{\micro\second} & N \\
       \midrule
       0.5         & 16 \\
       1           & 39 \\
       1.5         & 62 \\
       2           & 85 \\
       2.5         & 108 \\
       3           & 131 \\
       3.5         & 154 \\
       4           & 177 \\
       4.5         & 200 \\
       5           & 223 \\
       \bottomrule
   \end{tblr}
\end{table}
From these values a linear regression is performed to convert channel numbers to times. 
The fit yields 
\begin{equation*}
    t=(0.022\pm0)\,\unit{\micro\second}\cdot N - (0.152\pm0)\,\unit{\micro\second}.
\end{equation*}
where $t$ is the time and $N$ is the channel number. 
The measured data and linear fit are shown in \autoref{fig:calibration}. 
\begin{figure}
   \centering
   \includegraphics[width =0.7\textwidth]{build/Calibration.pdf}
   \caption{Linear regression of channel number versus delay time.}
   \label{fig:calibration}
\end{figure}
Using this calibration the measured spectrum, shown in \autoref{fig:spectrum_raw}, can be analyzed.
\begin{figure}
   \centering
   \includegraphics[width =0.7\textwidth]{build/Spectrum_raw.pdf}
   \caption{Measured lifetime spectrum of muons.}
   \label{fig:spectrum_raw}
\end{figure}
First the channel number is converted to time using the results from calibrating the MCA. 
Also for visualization purposes, the tail end of the spectrum is not shown in graphs, as there are only very few counts. 
However following calculations are still perfomored based on that data. 
The resulting spectrum is shown in \autoref{fig:spectrum_calibrated}.
\begin{figure}
   \centering
   \includegraphics[width =0.7\textwidth]{build/Spectrum.pdf}
   \caption{Calibrated lifetime spectrum of muons.}
   \label{fig:spectrum_calibrated}
\end{figure}
To determine the lifetime of muons from this spectrum, an exponential decay function with an offset is fitted to the data, 
\begin{equation*}
    N(t) = N_0 \exp\left(-\frac{t}{\tau}\right) + U_0,
\end{equation*}
where $N_0$ is the initial count rate, $\tau$ is the muon lifetime and $U_0$ is a constant offset accounting for background noise, 
e.g. from myons that are not kept track of. 
To increase the accuracy of the fit, the first few channels were excluded, as there were no counts. 
Furthermore some more channels were excluded because of a peak that deviates from exponential decay. 
The fit yields
\begin{align*}
    N_0 &= 273 \pm 15 \\
    \tau &= 1.7 \pm 0.1 \,\unit{\micro\second} \\
    U_0 &= -3 \pm 4 \,.
\end{align*}
The fitted curve is shown as logarithmic plot in \autoref{fig:spectrum_fit}.
\begin{figure}
   \centering
   \includegraphics[width =0.7\textwidth]{build/Spectrum_log.pdf}
   \caption{Fitted exponential decay to the muon lifetime spectrum.}
   \label{fig:spectrum_fit}
\end{figure}
The blue area marks the channels that were neglected due to low count rates, while the red area was neglected due to the peak. 
% templates

%\begin{table}
%    \centering
%    \caption{.}
%    \label{tab:}
%    \sisetup{table-format = 3.0} 
%    \begin{tblr}{
%        colspec = {S S S},
%        row{1} = {guard, mode = math}
%    }
%        \toprule 
%        x \mathbin{/} \unit{\milli\meter} \\
%        \midrule
%        %data here
%        \bottomrule
%    \end{tblr}
%\end{table}

%\begin{figure}
%    \centering
%    \includegraphics[width =0.7\textwidth]{}
%    \caption{.}
%    \label{fig:}
%\end{figure}

%\begin{wrapfigure}[20]{r}{0.5\textwidth}
%    \begin{center}
%        \includegraphics[width =0.7\textwidth]{figures/}
%        \caption{}
%        \label{fig:}
%    \end{center}
%\end{wrapfigure}
\section{Theory}
\label{sec:theory}
\subsection{Muons}
Muons are charged leptons which have much greater mass of $\tau_{\mu}=\qty{105.6583745\pm0.0000024}{\mega\electronvolt}$ \cite{PDG} compared to electrons. 
Also they are unstable and decay via
\begin{equation}
    \ce{\mu^- -> e^- + \bar{\nu}_e + \nu_\mu} 
\end{equation}
following poissonian statistics. 
Analogously the anti muon decays via
\begin{equation}
    \ce{\mu^+ -> e^+ + \nu_e + \bar{\nu}_\mu}.
\end{equation}
Their lifetime is approximately $\tau_{\mu}=\qty{2.1969811\pm0.000002}{\micro\second}$ \cite{PDG}. 
The lifetime $\tau$ is defined via the decay law
\begin{equation}
    N(t) = N_0 \exp\left(-\frac{t}{\tau}\right).
\end{equation}  
Cosmic muons are created in the upper atmosphere at around $10\,\unit{\kilo\meter}$ \cite{Leifi} by $ \pi^+ \rightarrow \mu^+ + \nu_\mu $ 
and analogously with conjugated charges for negative muons. 
The pions themselves are created by high energy cosmic rays colliding with nuclei in the atmosphere.
The muons move with a speed close to the speed of light which causes time dilation. 
Therefore their range in the atmosphere is extended, compared to non relativstic expectations such that they can reach earths surface. 
\subsection{Scintillation Detectors} %%% Source
A scintillation detector is made up of a scintillator material and a photomultiplier tube. 
The scintillator converts the energy of incoming charged particles into photons, which in turn are converted into 
photoelectrons by the photomultiplier tube. 
In linear regime the number of photoelectrons is proportional to the energy deposited by the particle in the scintillator.
The photomultiplier tubes work via a cascade of electrons which is created by secondary emission on dynodes.
Dynodes have a low work function, so that incoming electrons can knock out multiple secondary electrons, but suffer from 
strong thermal noise. \cite{leo}

\section{Theory}
\label{sec:theory}
\subsection{Muons}
Muons are charged leptons which have much greater mass of VALUE compared to electrons. %%% Value
Also they are unstable and decay via
\begin{equation}
    \ce{\mu^- -> e^- + \bar{\nu}_e + \nu_\mu}.
\end{equation}
Analogously the anti muon decays via
\begin{equation}
    \ce{\mu^+ -> e^+ + \nu_e + \bar{\nu}_\mu}.
\end{equation}
Their lifetime is approximately VALUE. %%% Value
The lifetime $\tau$ is defined via the decay law
\begin{equation}
    N(t) = N_0 \exp\left(-\frac{t}{\tau}\right).
\end{equation}  
%%% Add something?
Cosmic muons are created in the upper atmosphere at around $10-20\,\unit{\kilo\meter}$ by PION DECAY and %%% Decay
move with a speed close to the speed of light (ENERGY). %%% Source and Value
Due to relativistic effects their range is not only $600\,\unit{\meter}$ as would be expected classically, but 
around $60\,\unit{\kilo\meter}$ instead, which allows them to reach earths surface. %%% Source
\subsection{Scintillation Detectors} %%% Source
A scintillation detector is made up of a scintillator material and a photomultiplier tube. 
The scintillator converts the energy of incoming charged particles into photons, which in turn are converted into 
photoelectrons by the photomultiplier tube. 
In linear regime the number of photoelectrons is proportional to the energy deposited by the particle in the scintillator.
The photomultiplier tubes work via a cascade of electrons which is created by secondary emission on dynodes.
Dynodes have a low work function, so that incoming electrons can knock out multiple secondary electrons, but suffer from 
strong thermal noise. 

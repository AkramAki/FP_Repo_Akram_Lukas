\section{Experimental Setup and Procedure}
\label{sec:procedure}
In the following, the experimental setup of the Helium-Neon laser resonator is described, followed by the complete procedure 
performed in this work.


\subsection{Procedure}
The experiment was performed on a linear optical rail as shown in \autoref{fig:experimental_setup}. 
All optical components were mounted on this rail with adjustable holders. 
The HeNe laser tube used had a length of $l = 408 \,\mathrm{mm}$ and was filled with a He:Ne gas mixture at a ratio 5:1. 
A pre-adjusted alignment laser ($\lambda = 532\,\mathrm{nm}$, $P \approx 0.1\,\mathrm{mW}$) was used for the initial coarse alignment of
the resonator axis. Two resonator mirrors were available: (i) one concave output coupler with radius of curvature $r = 1400 \,\mathrm{mm}$ and a 
transmission $T \approx 1.5..1.8\%$ and (ii) either a planar high-reflectivity mirror or a second concave high-reflectivity mirror 
($r = 1400 \,\mathrm{mm}$, $R \ge 99\%$). Two alignment apertures with crosshairs were placed at maximum distance on the rail.

\begin{figure}[h]
    \centering
    \includegraphics[width = 1\textwidth]{content/Images/experimental_setup.png}
    \caption{Experimental setup of the resonator on an optical rail. Image adapted from \cite{Versuchsanleitung}.}
    \label{fig:experimental_setup}
\end{figure}

After alignment, the internal discharge of the HeNe tube was driven at $I = 6.5\,\mathrm{mA}$ (as recommended).

\subsection{Alignment}
Before any measurements can be performed, the optical axis of the resonator must be 
established. Therefore, a precise alignment of all components on the optical rail is required until stable lasing is achieved.
The alignment was done in the following order:
\begin{enumerate}
    \item alignment laser on rail, apertures placed at maximum rail separation,
    \item alignment laser beam centered on both crosshairs,
    \item placement of the two resonator mirrors and the discharge tube such that the backreflection returns to the center of the rear aperture,
    \item switch off the alignment laser, apply the discharge current, and fine tune mirror angular orientation until stable lasing occurred.
\end{enumerate}

\begin{figure}[h]
    \centering
    \includegraphics[width =1\textwidth]{content/Images/alignment.png}
    \caption{Alignment procedure for the Helium-Neon Laser. Image adapted from \cite{Versuchsanleitung}.}
\end{figure}


\subsection{Measurement Tasks}
After alignment, the following tasks were performed.  
For each task it is explicitly stated which data points must be recorded.

\subsubsection{Verification of Resonator Stability}

Two resonator configurations were investigated: planar--concave and concave--concave ($r = 1400\,\mathrm{mm}$ each).  
For each configuration:
\begin{enumerate}
    \item maximize output power with photodiode
    \item increase mirror separation $L$ stepwise
\end{enumerate}

For every step of $L$ check whether stable laser activity can be observed. Repeat until stability is not 
achievable and record the length for later comparison against the theoretical value.

\subsubsection{Observation of TEM Modes}
A tungsten wire of diameter $d = \qty{0.005}{\milli\meter}$ was inserted into the resonator between tube and 
output coupler. First, the mode structure was visually inspected on a screen (qualitative identification). 
To enlarge the image on the screen and make the mode structure more clearly distinguishable, a lens 
was placed in front of the screen. The tungsten wire was then used once as a mode stop: it was first positioned centrally in 
the beam to block the central intensity maximum of the TEM\textsubscript{00} mode, such that a TEM\textsubscript{01}-like mode 
became dominant and could be observed. Afterwards, the wire was moved once more to a position where it blocked one of 
the two lateral maxima of this TEM\textsubscript{01}-like mode, thereby suppressing this higher mode and 
allowing the TEM\textsubscript{00} mode to be isolated again for measurement.
Afterwards, the screen was removed and replaced by the photodiode. For the quantitative measurement, 
the photodiode was moved horizontally across the beam to record the transverse intensity distribution.
For each measurement the mode index (e.g.\ TEM$_{00}$ or TEM$_{01}$) was noted, and the photodiode voltage $U(x)$ was 
recorded as a function of the horizontal detector position $x$.


\subsubsection{Determination of Polarization}

A polarizer was placed after the output coupler.  
The polarizer was rotated in steps of $\qty{18}{\deg}$.
For each rotation angle $\phi$ record:
\begin{itemize}
    \item $\phi$ in degrees
    \item photodiode power $P(\phi)$
\end{itemize}

The resulting $P(\phi)$ curve was later compared to the theoretical prediction given by Malus’ law.

\subsubsection{Multimode Operation and Beat Frequencies}

Without inserting a Fabry--Perot etalon, the laser operated in multimode regime, i.e.\ several longitudinal modes oscillated simultaneously. 
To investigate this, a fast photodiode with a bandwidth of up to $\qty{1}{\giga\hertz}$ was placed behind the output coupler. 
Its signal was recorded with a spectrum analyser. 
The measurement was repeated for multiple resonator lengths $L$. For each value of $L$ the Fourier spectrum was acquired and the corresponding beat 
frequencies were extracted. 
For each resonator length $L$ (in $\unit{\milli\meter}$) the beat frequencies $f_{\mathrm{beat}}(L)$ corresponding to the peaks in the Fourier 
spectrum were recorded.


\subsubsection{Determination of the Laser Wavelength}

The wavelength of the He--Ne laser was determined using a transmission grating with a grating constant of 
$\qty{80}{\per\milli\meter}$. The laser beam was directed onto the grating, and the resulting diffraction pattern was projected onto 
a screen at a distance $\qty{85.2(0.2)}{\milli\meter}$. The horizontal positions $x_n$ of the $n$-th order maxima were measured, 
and from these the wavelength was determined using the grating equation.
For each measurement the screen distance $D$ and the horizontal positions $x_n$ of the diffraction maxima were recorded.

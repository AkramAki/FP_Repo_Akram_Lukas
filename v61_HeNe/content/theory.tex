\section{Theory} %% Sources and pictures
\label{sec:theory}
This section covers the theoretical background needed to describe the measurements described in \autoref{sec:procedure}, 
i.\;e. the working principle of lasers, their amplification, stability condition 
and descriptions of polarization, longitudinal and transverse modes as well as interference at a grating. 

\subsection{Working principle of lasers}
\label{subsec:principle}
Lasers consist of three main parts: The pump, an active medium and a resonator. 
The pump is used to create an population inversion in the active medium. 
The coherent laser light is then created by stimulated emission in the active medium. 
Since an population inversion is needed in order to amplify the emission, 
at least a three level system is needed because the population inversion would counteract against the pump. 
To create the amplification the resonator used. 
A resonator is a set of two mirrors, one completely reflective and one partially transmittive, which allows light with the wavelength $\lambda=n\cdot L/2$ to be amplified. 
This resonance condition is equivalent to $f=\frac{cL}{2n}$ with $f$ being the light frequency. 
A resonator also gives rise to many different modes, which are described by different order of $n\in \mathbb{N}$ and equdistant in frequency. 
Multiple of these modes can be excited if the spectral envelope, created by the linewidth of the laser transition in the active medium. 

In order for a laser to be stable, the stability condition 
\begin{equation*}
    0\leq(1-\frac{L}{R_1})(1-\frac{L}{R_2})\leq1
\end{equation*}
with $R_{\symup{i}}$ being the mirrors curvature radius. 
If both mirrors are flat, the expression always equals $1$. 
If only one of the mirrors is flat the expressions is equal to $(1-\frac{L}{r})^2$, therefore the maximum length of the resonator is $L=2r$.  

The amplification can be described as
\begin{equation*}
    G(f,z)=\frac{I(f,z)}{I(f,0)}=\exp(-\alpha(f)z)
\end{equation*}
where $\alpha$ is the population and freqeuncy dependent absorption
\begin{equation*}
    \alpha(f)=(N_1-\frac{g_1}{g_2}N_2)\sigma(f)
\end{equation*}
with the absorption cross section $\sigma$. 
When considering losses with a rate of $\gamma$, the amplification condition requires the population inversion to be
\begin{equation*}
    \Delta N>\frac{\gamma}{2\sigma L}
\end{equation*}
with $L$ being the medium length. 

\subsection{Characteristics of a helium-neon-laser}
The helium-neon configuration is used as a four level system to create coherent red light ($\lambda=633\,\unit{\nm}$) using the $3s_2\rightarrow2p_4$ transition in neon. 
The $3s_2$ state is populated by pumping Helium into an excited state via gas discharge that is caused by a current applied on the medium 
and collisions transferring energy from excited helium atoms to neon in order to populate the $3s_2$ neon state. 
The spectral width of the transition is increased by approximately $1500\,\unit{\mega\hertz}$, because the neon atoms are moving, which causes a doppler shift. 
The broadening is a result of the statistical distribution of velocities in the medium, therefore this effect is called doppler broadening. 

\subsection{Transverse modes}
\label{subsec:trans_modes}
Transverse modes are intensity profiles of the beam perpendicular to the direction of propagation and usually called TEM-modes. %double check
Slicing through the profile of the corresponding electric field in $x$-direction yields
\begin{equation*}
E_{\symup{m}}(x)=H_{\symup{m}}\left(\frac{\sqrt{2}}{w_0}x\right)\exp\left(-\left(\frac{\sqrt{2}}{4w_0}x\right)^2\right)
\end{equation*}
with $H_{\symup{m}}$ as the hermite-polynomials and $w_0$ as the beam width.

\subsection{Interference at a grating}
\label{subsec:interference}
Coherent light with wavelength $\lambda$ can be diffracted at a grating with grating constant $g$. 
The resulting pattern on a screen with distance $l$ to the grating forms sharp interference maxima, which might be described as
\begin{equation*}
    \lambda=\frac{g}{k}\sin\left(\arctan\left(\frac{a_{\symup{k}}}{l}\right)\right)
\end{equation*}
with $a_{\symup{k}}$ being the distance between the maxima of order zero and order $k$ on the screen.

\section{Analysis} %% Change [] for units to /
\label{sec:analysis}

\subsection{Stability}
By changing the distance between the mirrors the stability condition, 
which was described in \autoref{subsec:principle}, could be tested.
When using mirrors with a radius of curvature of $R=1400\,\unit{\mm}$ on both ends no loss of stability was observable, even for the 
longest distance of $L=(217.2\pm0.2)\,\unit{\cm}$. 
When using a flat mirror on one end and keeping the same curved mirror on the other, 
stability was lost at a distance of $L=(88,6\pm0.2)\,\unit{\cm}$.

\subsection{Polarization}
The power was measured as a function of polarization angle and is documented in \autoref{tab:polarization}. % double check name
\begin{table}
   \centering
   \caption{Power $P$ for varying polarization angle $\theta$.}
   \label{tab:polarization}
   \sisetup{table-format = 3.0} 
   \begin{tblr}{
       colspec = {S S[table-format=1.3] S[table-format=1.3]},
       row{1} = {guard, mode = math},
       vline{3}={2}{-}{text=\clap{$\pm$}},
    }
       \toprule 
       \theta \mathbin{/} \unit{\degree} &\SetCell[c=2]{c} P \mathbin{/} \unit{\milli\watt}\\
       \midrule
       %data here
        0	& 0.994   & 0.001   \\
        18	& 0.188   & 0.001   \\
        36	& 0.038   & 0.001   \\
        54	& 0.616   & 0.001   \\
        72	& 1.778   & 0.005   \\
        90	& 2.915   & 0.003   \\
        108	& 3.82    & 0.01    \\
        126	& 3.83    & 0.05    \\
        144	& 3.39    & 0.05    \\
        162	& 2.19    & 0.03    \\
        180	& 1.03    & 0.001   \\
        198	& 0.201   & 0.001   \\
        216	& 0.034   & 0.001   \\
        234	& 0.636   & 0.001   \\
        252	& 1.81    & 0.003   \\
        270	& 2.985   & 0.005   \\
        288	& 3.83    & 0.005   \\
        306	& 3.957   & 0.005   \\
        324	& 3.31    & 0.01    \\
        342	& 2.08    & 0.03    \\
        360	& 1.01    & 0.002   \\
       \bottomrule
   \end{tblr}
\end{table}
The same data is visualized in \autoref{fig:polarization}, together with a fit using the law of malus described in 
\autoref{sec:theory}. % update to subsec?
\begin{figure}
   \centering
   \includegraphics[width =0.7\textwidth]{build/Polarisation.pdf}
   \caption{Polarization measured as Power $P$ as a function of polarizing angle $\theta$.}
   \label{fig:polarization}
\end{figure}
The fit yields 
\begin{align*}
    P_0&=(3.97\pm0.02)\,\unit{milli\watt} \\
    \theta_0&=(120.44\pm0.06)\,\unit{\degree}\;. \\
\end{align*}

\subsection{Transverse modes}
Both the TEM00- and the TEM01-modes were observable and measured along $x$-direction. 
The corresponding data is documented in \autoref{tab:TEM}. 

\begin{table}
   \centering
   \caption{Measurement of TEM00- and TEM01-modes as diode current $I$ along the $x$-direction.}
   \label{tab:TEM}
   \sisetup{table-format = 3.0} 
   \begin{tblr}{
       colspec = {S S S S S},
       row{1} = {guard, mode = math},
       vline{3,5}={2}{-}{text=\clap{$\pm$}},
   }
       \toprule 
    %    & SetCell[c=2]{c} \text{TEM00-mode} && SetCell[c=2]{c} \text{TEM01-mode} \\
       x \mathbin{/} \unit{\milli\meter} &\SetCell[c=2]{c} I_{00} \mathbin{/} \unit{\micro\ampere} 
       &&\SetCell[c=2]{c} I_{01} \mathbin{/} \unit{\micro\ampere} \\
       \midrule
       %data here
        -30	& 0.009	& 0.0005 &&&\\
        -29	& 0.013	& 0.0005 &&&\\
        -28	& 0.014	& 0.0005 &&&\\
        -27	& 0.0215& 0.0005 &&&\\
        -26	& 0.017	& 0.0005 &&&\\
        -25	& 0.018	& 0.0005 &&&\\
        -24	& 0.0322& 0.0007 &&&\\
        -23	& 0.062	& 0.001  &&&\\
        -22	& 0.113	& 0.001  &&&\\
        -21	& 0.131	& 0.002  &&&\\
        -20	& 0.145	& 0.004  &&&\\
        -19	& 0.22	& 0.005  &&&\\
        -18	& 0.38	& 0.005  &&&\\
        -17	& 0.63	& 0.01   &&&\\
        -16	& 0.98	& 0.02   &&&\\
        -15	& 1.18	& 0.04   &&&\\
        -14	& 1.59	& 0.04   &&&\\
        -13	& 2.33	& 0.05   &&&\\
        -12	& 3.10	& 0.05   &&&\\
        -11	& 3.62	& 0.05   &&&\\
        -10	& 3.95	& 0.05   &&&\\
        -9	& 4.10	& 0.03   &&&\\
        -8	& 4.40	& 0.03   &&&\\
        -7	& 4.58	& 0.03   &&&\\
        -6	& 4.48	& 0.03   &&&\\
        -5	& 4.62	& 0.03   &&&\\
        -4	& 4.60	& 0.03   &&&\\
        -3	& 4.52	& 0.03   &&&\\
        -2	& 4.25	& 0.03   &&&\\	
        -1	& 3.60	& 0.03   &&&\\
        0	& 3.40	& 0.03   &&&\\
        1	& 2.98	& 0.03   &&&\\
        2	& 2.62	& 0.03   &&&\\
        3	& 1.82	& 0.03   &&&\\
        4	& 1.65	& 0.01   &&&\\
        5	& 1.40	& 0.02   &&&\\
        6	& 0.81	& 0.02   &&&\\		
        7	& 0.59	& 0.01   &&&\\
        8	& 0.34	& 0.01   &&&\\
        9	& 0.25	& 0.01   &&&\\
        10	& 0.200	& 0.02   &&&\\	
        11	& 0.138	& 0.02   &&&\\
        12	& 0.066	& 0.01   &&&\\
        13	& 0.041	& 0.01   &&&\\
        14	& 0.0355& 0.0005 &&&\\
        15	& 0.0155& 0.0005 &&&\\
        16	& 0.0222& 0.0005 &&&\\
        17	& 0.0065& 0.0005 &&&\\
        18	& 0.0055& 0.0001 &&&\\
        19	& 0.0053& 0.0001 &&&\\
        20	& 0.0034& 0.0001 &&&\\
        20	& 0.0034& 0.0001 &&&\\
        20	& 0.0034& 0.0001 &&&\\
       \bottomrule
   \end{tblr}
\end{table}
The TEM00-mode is plotted in \autoref{fig:TEM00} and 
fitted according to the zeroth order mode discussed in \autoref{subsec:trans_modes}. 

\begin{figure}
   \centering
   \includegraphics[width =0.7\textwidth]{build/TEM00.pdf}
   \caption{Diode current $I$ with respect to position on $x$ axis, fitted with the TEM00-mode.}
   \label{fig:TEM00}
\end{figure}

The TEM01-mode is plotted and fitted analogously in \autoref{fig:TEM01}
\begin{figure}
   \centering
   \includegraphics[width =0.7\textwidth]{build/TEM00.pdf}
   \caption{Diode current $I$ with respect to position on $x$ axis, fitted with the TEM00-mode.}
   \label{fig:TEM01}
\end{figure}

% templates

%\begin{table}
%    \centering
%    \caption{.}
%    \label{tab:}
%    \sisetup{table-format = 3.0} 
%    \begin{tblr}{
%        colspec = {S S S},
%        row{1} = {guard, mode = math}
%    }
%        \toprule 
%        x \mathbin{/} \unit{\milli\meter} \\
%        \midrule
%        %data here
%        \bottomrule
%    \end{tblr}
%\end{table}


%\begin{wrapfigure}[20]{r}{0.5\textwidth}
%    \begin{center}
%        \includegraphics[width =0.48\textwidth]{figures/}
%        \caption{}
%        \label{fig:}
%    \end{center}
%\end{wrapfigure}
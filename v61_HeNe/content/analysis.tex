\section{Analysis} %% Change [] for units to /
\label{sec:analysis}

\subsection{Stability}
By changing the distance between the mirrors the stability condition, 
which was described in \autoref{subsec:principle}, could be tested.
When using mirrors with a radius of curvature of $R=1400\,\unit{\mm}$ on both ends no loss of stability was observable, even for the 
longest distance of $L=(217.2\pm0.2)\,\unit{\cm}$. 
When using a flat mirror on one end and keeping the same curved mirror on the other, 
stability was lost at a distance of $L=(88.6\pm0.2)\,\unit{\cm}$.

\subsection{Polarization}
The power was measured as a function of polarization angle and is documented in \autoref{tab:polarization}. % double check name
\begin{table}
   \centering
   \caption{Power $P$ for varying polarization angle $\theta$.}
   \label{tab:polarization}
   \sisetup{table-format = 3.0} 
   \begin{tblr}{
      colspec = {S S[table-format=1.3] S[table-format=1.3]},
      row{1} = {guard, mode = math},
      vline{3}={2}{-}{text=\clap{$\pm$}},
      }
       \toprule 
       \theta \mathbin{/} \unit{\degree} &\SetCell[c=2]{c} P \mathbin{/} \unit{\milli\watt}\\
       \midrule
       %data here
       0	& 0.994   & 0.001   \\
       18	& 0.188   & 0.001   \\
       36	& 0.038   & 0.001   \\
       54	& 0.616   & 0.001   \\
       72	& 1.778   & 0.005   \\
       90	& 2.915   & 0.003   \\
       108	& 3.82    & 0.01    \\
       126	& 3.83    & 0.05    \\
       144	& 3.39    & 0.05    \\
       162	& 2.19    & 0.03    \\
       180	& 1.03    & 0.001   \\
       198	& 0.201   & 0.001   \\
       216	& 0.034   & 0.001   \\
       234	& 0.636   & 0.001   \\
       252	& 1.81    & 0.003   \\
       270	& 2.985   & 0.005   \\
       288	& 3.83    & 0.005   \\
       306	& 3.957   & 0.005   \\
       324	& 3.31    & 0.01    \\
       342	& 2.08    & 0.03    \\
       360	& 1.01    & 0.002   \\
       \bottomrule
      \end{tblr}
   \end{table}
   The same data is visualized in \autoref{fig:polarization}, together with a fit using the law of malus described in 
   \autoref{sec:theory}. % update to subsec?
   \begin{figure}
      \centering
      \includegraphics[width =0.7\textwidth]{build/Polarisation.pdf}
      \caption{Polarization measured as Power $P$ with varying polarization angle $\theta$.}
      \label{fig:polarization}
   \end{figure}
   The fit yields 
   \begin{align}
      \begin{split}
         P_0&=(3.97\pm0.02)\,\unit{\milli\watt} \\
         \theta_0&=(120.44\pm0.06)\,\unit{\degree}\;. \\
         \label{eq:polarization_fit}
      \end{split}
   \end{align}
   
   \subsection{Transverse modes}
   Both the TEM00- and the TEM01-modes were observable and measured along $x$-direction. 
   The corresponding data is documented in \autoref{tab:TEM}. 
   
   \begin{longtblr}[
      entry={centering},
   caption={Measurement of TEM00- and TEM01-modes along $x$-direction as diode current $I$.},
   label={tab:TEM},
   ]
   % \begin{tblr}
      {
       colspec = {S[table-format=4.0] S[table-format=1.4] S[table-format=4.4] S[table-format=1.4] S[table-format=1.4]},
       row{1} = {guard, mode = math},
       vline{3,5}={2}{-}{text=\clap{$\pm$}},
   }
       \toprule 
       x \mathbin{/} \unit{\milli\meter} &\SetCell[c=2]{c} I_{00} \mathbin{/} \unit{\micro\ampere} 
       &&\SetCell[c=2]{c} I_{01} \mathbin{/} \unit{\micro\ampere} \\
       \midrule
       %data here
        -30	& 0.009	& 0.0005 & 0.0011	& 0.0001\\
        -29	& 0.013	& 0.0005 & 0.0032	& 0.0003\\
        -28	& 0.014	& 0.0005 & 0.006 	& 0.001\\
        -27	& 0.0215 & 0.0005 & 0.012	& 0.001\\
        -26	& 0.017	& 0.0005 & 0.007	& 0.0005\\
        -25	& 0.018	& 0.0005 & 0.022	& 0.0005\\
        -24	& 0.0322 & 0.0007 & 0.065	& 0.005\\
        -23	& 0.062	& 0.001  & 0.110	& 0.010\\
        -22	& 0.113	& 0.001  & 0.120	& 0.005\\
        -21	& 0.131	& 0.002  & 0.170	& 0.015\\
        -20	& 0.145	& 0.004  & 0.3	   & 0.05\\
        -19	& 0.22	& 0.005  & 0.54	& 0.05\\
        -18	& 0.38	& 0.005  & 0.7	   & 0.02\\
        -17	& 0.63	& 0.01   & 0.09	& 0.05\\
        -16	& 0.98	& 0.02   & 1.04	& 0.05\\
        -15	& 1.18	& 0.04   & 1.15	& 0.05\\
        -14	& 1.59	& 0.04   & 1.27	& 0.08\\
        -13	& 2.33	& 0.05   & 1.45	& 0.05\\
        -12	& 3.10	& 0.05   & 1.35	& 0.05\\
        -11	& 3.62	& 0.05   & 1.2	   & 0.05\\
        -10	& 3.95	& 0.05   & 0.95	& 0.05\\
        -9	& 4.10	& 0.03   & 0.7	   & 0.05\\
        -8	& 4.40	& 0.03   & 0.55	& 0.05\\
        -7	& 4.58	& 0.03   & 0.03	& 0.03\\
        -6	& 4.48	& 0.03   & 0.105	& 0.005\\
        -5	& 4.62	& 0.03   & 0.011	& 0.001\\
        -4	& 4.60	& 0.03   & 0.040	& 0.005\\
        -3	& 4.52	& 0.03   & 0.13	& 0.01\\
        -2	& 4.25	& 0.03   & 0.2	   & 0.05\\	
        -1	& 3.60	& 0.03   & 0.32	& 0.03\\
        0	& 3.40	& 0.03   & 0.4	   & 0.05\\
        1	& 2.98	& 0.03   & 0.43	& 0.05\\
        2	& 2.62	& 0.03   & 0.58	& 0.03\\
        3	& 1.82	& 0.03   & 0.5	   & 0.05\\
        4	& 1.65	& 0.01   & 0.68	& 0.08\\
        5	& 1.40	& 0.02   & 0.85	& 0.02\\
        6	& 0.81	& 0.02   & 0.7	   & 0.05\\		
        7	& 0.59	& 0.01   & 0.62	& 0.05\\
        8	& 0.34	& 0.01   & 0.48	& 0.03\\
        9	& 0.25	& 0.01   & 0.43	& 0.03\\
        10	& 0.20	& 0.02   & 0.31	& 0.03\\	
        11	& 0.14	& 0.02   & 0.22	& 0.02\\
        12	& 0.07	& 0.01   & 0.105	& 0.003\\
        13	& 0.04	& 0.01   & 0.110	& 0.005\\
        14	& 0.0355& 0.0005 &  0.100	& 0.005\\
        15	& 0.0155& 0.0005 &  0.057	& 0.003\\
        16	& 0.0222& 0.0005 &  0.038	& 0.003\\
        17	& 0.0065& 0.0005 &  0.015	& 0.002 \\
        18	& 0.0055& 0.0001 &  0.010	& 0.002 \\
        19	& 0.0053& 0.0001 &  0.0082	& 0.0003\\
        20	& 0.0034& 0.0001 &  0.0027	& 0.0003\\
       \bottomrule
   % \end{tblr}
\end{longtblr}
The TEM00-mode is plotted in \autoref{fig:TEM00} and 
fitted according to the zeroth order mode in equation \eqref{eq:t_modes}, where the measured current is $I=I_0\cdot E^2$.  
\begin{figure}
   \centering
   \includegraphics[width =0.7\textwidth]{build/TEM00.pdf}
   \caption{Diode current $I$ with respect to position on $x$ axis, fitted with the TEM00-mode.}
   \label{fig:TEM00}
\end{figure}
The approximated function uses the parameters
\begin{align}
   \begin{split}
      I_0 &= (4.7 \pm 0.1)\,\unit{\micro\ampere}\\
      x_0 &= -(4.9 \pm 0.1)\,\unit{\mm}\\
      w_0 &= (12.24 \pm 0.09)\,\unit{\mm}\;.
      \label{eq:TEM00_fit}
   \end{split}
\end{align}

The TEM01-mode is plotted and fitted analogously in \autoref{fig:TEM01}.



\begin{figure}
   \centering
   \includegraphics[width =0.7\textwidth]{build/TEM01.pdf}
   \caption{Diode current $I$ with respect to position on $x$ axis, fitted with the TEM01-mode.}
   \label{fig:TEM01}
\end{figure}
This fit yields
\begin{align}
   \begin{split}
   I_0 &= (0.63 \pm 0.05)\,\unit{\micro\ampere}\\
   x_0 &= -(4.5 \pm 0.1)\,\unit{\mm}\\
   w_0 &= (11.3 \pm 0.1)\,\unit{\mm}\;.
   \label{eq:TEM01_fit}
   \end{split}
\end{align}

\subsection{Determining the wavelength}
The wavelength can be determined via the interference pattern as described in \autoref{subsec:interference}, 
using the data acquired and displayed in \autoref{tab:grating}. 
The uncertainty of the measured positions $x$ is $\Delta x=0.2\,\unit{\cm}$
Using equation \eqref{eq:wavelength} this yields a mean wavelength of $\lambda=(640\pm4)\,\unit{\nm}$.
\begin{table}
   \centering
   \caption{Positions $x$ of interference maxima on the screen.}
   \label{tab:grating}
   \sisetup{table-format = 1.0} 
   \begin{tblr}{
       colspec = {S S[table-format=2.1] },
       row{1} = {guard, mode = math},
    }
       \toprule 
       \text{Order} & x \mathbin{/} \unit{\cm}\\
       \midrule
       %data here
         1	& 4.5 \\
         2	& 8.8 \\
         3	& 13.2 \\
         4	& 17.7 \\
         5	& 22.4 \\
         6	& 27.7 \\
         7	& 32.2 \\
         8	& 37.8 \\
       \bottomrule
   \end{tblr}
\end{table}

% templates

%\begin{table}
%    \centering
%    \caption{.}
%    \label{tab:}
%    \sisetup{table-format = 3.0} 
%    \begin{tblr}{
%        colspec = {S S S},
%        row{1} = {guard, mode = math}
%    }
%        \toprule 
%        x \mathbin{/} \unit{\milli\meter} \\
%        \midrule
%        %data here
%        \bottomrule
%    \end{tblr}
%\end{table}


%\begin{wrapfigure}[20]{r}{0.5\textwidth}
%    \begin{center}
%        \includegraphics[width =0.48\textwidth]{figures/}
%        \caption{}
%        \label{fig:}
%    \end{center}
%\end{wrapfigure}
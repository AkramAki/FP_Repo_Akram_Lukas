\section{Analysis} %% Change [] for units to /
\label{sec:analysis}

\subsection{Stability}
By changing the distance between the mirrors the stability condition, 
which was described in \autoref{subsec:principle}, could be tested.
When using mirrors with a radius of curvature of $R=1400\,\unit{\mm}$ on both ends no loss of stability was observable, even for the 
longest distance of $L=(217.2\pm0.2)\,\unit{\cm}$. 
When using a flat mirror on one end and keeping the same curved mirror on the other, 
stability was lost at a distance of $L=(88,6\pm0.2)\,\unit{\cm}$.

\subsection{Polarization}
The power was measured as a function of polarization angle and is documented in \autoref{tab:polarization}. % double check name
\begin{table}
   \centering
   \caption{Power $P$ for varying polarization angle $\theta$.}
   \label{tab:polarization}
   \sisetup{table-format = 3.0} 
   \begin{tblr}{
       colspec = {S S[table-format=1.3] S[table-format=1.3]},
       row{1} = {guard, mode = math},
       vline{3}={2}{-}{text=\clap{$\pm$}},
    }
       \toprule 
       \theta \mathbin{/} \unit{\degree} &\SetCell[c=2]{c} P \mathbin{/} \unit{\milli\watt}\\
       \midrule
       %data here
        0	& 0.994   & 0.001   \\
        18	& 0.188   & 0.001   \\
        36	& 0.038   & 0.001   \\
        54	& 0.616   & 0.001   \\
        72	& 1.778   & 0.005   \\
        90	& 2.915   & 0.003   \\
        108	& 3.815   & 0.01    \\
        126	& 3.83    & 0.05    \\
        144	& 3.387   & 0.05    \\
        162	& 2.188   & 0.03    \\
        180	& 1.03    & 0.001   \\
        198	& 0.201   & 0.001   \\
        216	& 0.034   & 0.001   \\
        234	& 0.636   & 0.001   \\
        252	& 1.81    & 0.003   \\
        270	& 2.985   & 0.005   \\
        288	& 3.83    & 0.005   \\
        306	& 3.957   & 0.005   \\
        324	& 3.31    & 0.01    \\
        342	& 2.083   & 0.03    \\
        360	& 1.01    & 0.002   \\
       \bottomrule
   \end{tblr}
\end{table}
The same data is visualized in \autoref{fig:polarization}, together with a fit using the law of malus described in 
\autoref{sec:theory}. % update to subsec?
\begin{figure}
   \centering
   \includegraphics[width =0.48\textwidth]{build/Polarisation.pdf}
   \caption{.}
   \label{fig:polarization}
\end{figure}
The fit yields 
\begin{align*}
    P_0&=(3.97\pm0.02)\,\unit{milli\watt} \\
    \theta_0&=(120.44\pm0.06)\,\unit{\degree}\;. \\
\end{align*}

\subsection{Transverse modes}
Both the TEM00- and the TEM01-modes were observable and measured along $x$-direction. 
The corresponding data is documented in \autoref{tab:TEM}. 

% \begin{table}
%    \centering
%    \caption{Measurement of TEM00- and TEM01-modes as diode current $I$ along the $x$-direction.}
%    \label{tab:}
%    \sisetup{table-format = 3.0} 
%    \begin{tblr}{
%        colspec = {S S S},
%        row{1} = {guard, mode = math}
%    }
%        \toprule 
%        & SetCell[c=2]{c} \text{TEM00-mode} && SetCell[c=2]{c} \text{TEM01-mode}
%        x \mathbin{/} \unit{\milli\meter} &\SetCell[c=2]{c} I_{00} \mathbin{/} \unit{\micro\ampere} && &\SetCell[c=2]{c} I_{01} \mathbin{/} \unit{\micro\ampere} \\
%        \midrule
%        %data here
%        \bottomrule
%    \end{tblr}
% \end{table}

% templates

%\begin{table}
%    \centering
%    \caption{.}
%    \label{tab:}
%    \sisetup{table-format = 3.0} 
%    \begin{tblr}{
%        colspec = {S S S},
%        row{1} = {guard, mode = math}
%    }
%        \toprule 
%        x \mathbin{/} \unit{\milli\meter} \\
%        \midrule
%        %data here
%        \bottomrule
%    \end{tblr}
%\end{table}


%\begin{wrapfigure}[20]{r}{0.5\textwidth}
%    \begin{center}
%        \includegraphics[width =0.48\textwidth]{figures/}
%        \caption{}
%        \label{fig:}
%    \end{center}
%\end{wrapfigure}
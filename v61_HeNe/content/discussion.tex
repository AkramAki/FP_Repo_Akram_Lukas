\section{Discussion}
\label{sec:discussion}
\subsection{Stability}
As expected, no instability could be observed with the $R=1400\,\unit{\mm}$ mirrors on both ends, as the length of the setup is less than $2.4\,\unit{\m}$, 
which is the expected maximum length as explained in \autoref{subsec:principle}. 
But with one flat mirror the instability was reached much earlier, at $0.886\,\unit{\m}$ instead of $1.4\,\unit{\m}$, which might be due to bad adjustment of the laser. 
During measurements and movements of the mirrors the setup was moved out of its operating window multiple times and 
it was not possible to recover good adjustment with similar power as before. 

\subsection{Polarization}
The Polarization follows Malus' law and the measurements could be approximated by its $\cos^2$ dependence with high precision. 
The fit parameters were given in equation \eqref{eq:polarization_fit}. 

\subsection{Transverse modes}
Both the TEM00- and TEM01-mode could be observed and fitted with the respective slices along the $x$-axis. 
The TEM00 measurements are well described by the fit with the parameters listed in equation \eqref{eq:TEM00_fit}, 
even though the measurements seem to represent a slightly slanted gaussian function. 
However, the TEM01 measurements could not be described by the respective function from equation \eqref{eq:t_modes}, % References
as the height of the two peaks differs in the data. 
Therefore the fit parameters listed in equation \eqref{eq:TEM01_fit} include larger uncertainties. 
The difference in height might be explained by an asymmetrical placement of the wire inside of the beam, which leads to a larger suppression of one peak compared to the other. 
More unlikely, but also possible is that some power was lost while taking the measurement. 

\subsection{Determining the wavelength}
The wavelength $\lambda_{\text{exp}}=(640\pm4)\,\unit{\nm}$ that was calculated from the data 
slightly deviates from the real wavelength of $\lambda_{\text{He-Ne}}=633\,\unit{\nm}$. 
Possibly the distance from grating to screen was measured incorrectly, 
or the screen was placed at a small angle which distorts the interference pattern and changes the distance between maxima. 
\section{Theory}
\label{sec:theory}

In this experiment the temperature-dependent reorientation of electric dipoles in 
a doped ionic crystal is examined.  
When an external electric field is applied, defect-dipoles align; 
after switching the field off and heating the sample at a controlled rate, 
these dipoles relax back towards a random distribution.  
This relaxation generates a measurable depolarisation current, from which 
the activation energy $W$ and the characteristic relaxation time $\tau_0$ can be extracted.

% -------------------------------------------------------------
\subsection{Defects and Dipole Formation in Ionic Crystals}
Ionic crystals consist of two interpenetrating sublattices of positively and negatively 
charged ions arranged in a regular three-dimensional structure.  
In a real crystal, deviations from perfect periodicity occur naturally: 
vacancies, interstitials and substitutional impurities.
When an ion of different charge replaces a host ion, charge neutrality requires the 
formation of compensating point defects, often in the form of nearby vacancies.  
A dopant-vacancy pair behaves as an electric dipole because the two charges occupy 
different lattice sites separated by a fixed distance.
These dipoles are not static, both the dopant and the relevant vacancy can hop 
between local positions if sufficient thermal energy is available.  
This thermally activated hopping underlies the temperature dependence of the dipole dynamics.

\begin{figure}[h!]
    \centering
    \includegraphics[width =0.5\textwidth]{content/images/different_vacancys.png}
    \caption{Schematic illustration of a dopant ion and an associated vacancy forming a 
    microscopic electric dipole in an ionic crystal \cite{defect_dipole_fig}.}
\end{figure}
\newpage

% -------------------------------------------------------------
\subsection{Alignment of Dipoles in an External Field}
At sufficiently high temperatures and without an external field, the dipoles are randomly oriented.  
When an electric field $E$ is applied, transitions aligning the dipole with the field 
become energetically favorable, so the system develops a nonzero macroscopic polarisation $P$.
Because each reorientation requires overcoming the local potential barrier of the lattice, 
the rate of reorientation is thermally activated.  
This is described by the relaxation time
$$
    \tau(T) = \tau_0 \exp\!\left( \frac{W}{k_{\mathrm{B}} T} \right),
$$
where $W$ is the activation barrier and $\tau_0$ is the high-temperature limit of the relaxation time.

After the dipoles have been aligned at a temperature where $\tau(T)$ is short compared to the 
field application time, the sample is cooled so that the dipoles become effectively frozen in place.

% -------------------------------------------------------------
\subsection{Thermally Stimulated Depolarisation}
Once the frozen‐in dipoles are created, the external field is removed.  
Heating the sample again shortens the relaxation time 
and the dipoles begin to return to an equilibrium distribution of random orientations.  
This collective reorientation produces a transient depolarisation current.

Let $N(T)$ be the number of dipoles that still remain aligned at temperature $T$.  
Their decay follows
$$
    \frac{\mathrm{d} N}{\mathrm{d} t}
    = -\frac{N(T)}{\tau(T)}.
$$
Because the polarisation is proportional to the number of aligned dipoles,  
the depolarisation current is
$$
    I(T) = -\frac{\mathrm{d}P}{\mathrm{d}t}
         \propto \frac{N(T)}{\tau(T)}.
$$

In the experiment the sample is heated at a constant rate
$$
    b = \frac{\mathrm{d}T}{\mathrm{d}t},
$$
which allows one to express all temperature derivatives in terms of $T$ alone.

% -------------------------------------------------------------
\subsection{Determining the Activation Energy $W$}
Two standard approaches are applied to determine the activation energy.

\subsubsection{Method I: Low-Temperature Approximation}
At temperatures well below the current maximum, the integral contributions in the 
exact expression for $I(T)$ become negligible.  
The current reduces to an exponential dependence,
$$
    I(T) \propto \exp\!\left( -\frac{W}{k_{\mathrm{B}} T} \right),
$$
so that a plot of
$$
    \ln I(T) \quad \text{vs.} \quad \frac{1}{T}
$$
yields a straight line with slope $-W/k_{\mathrm{B}}$.

\subsubsection{Method II: Integral Method via Polarisation}
A second method avoids approximations by expressing the instantaneous relaxation time as
$$
    \tau(T) = \frac{1}{I(T)} \int_T^{T_\text{end}} I(T')\, \mathrm{d}T',
$$
using the constant heating rate $b$.  
Taking the logarithm gives
$$
    \ln \tau(T) = \ln \tau_0 + \frac{W}{k_{\mathrm{B}} T},
$$
so that plotting $\ln \tau$ against $1/T$ again yields a straight line with slope $W/k_{\mathrm{B}}$.

% -------------------------------------------------------------
\subsection{Characteristic Relaxation Time \texorpdfstring{\(\tau_0\)}{tau0}}
The temperature $T_\text{max}$ at which the depolarisation current reaches its maximum 
contains direct information about $\tau_0$.  
Using the condition for the extremum of the exact current expression one obtains a relation of the form
$$
    \tau_0 = f(T_\text{max}, W, b),
$$
where $b$ is the heating rate.  
After $W$ is known from one of the previous methods, this expression yields the 
characteristic microscopic hopping time of the dipoles.
\newpage

\section{Analysis} %% Mention that sign of current is flipped
\label{sec:analysis}
In this chapter values for the heating rate $b$, activation Energy $W$, characteristic depolarization time scale $\tau_0$ 
and the temperature dependence of the depolarization time $\tau(T)$ will be calculated from two datasets with different heating rates
according to the polarization and integration method derived in \autoref{sec:theory}. 
The data is shown in the appendix. 
\subsection{Heating rate $b_1=(1.69\pm0.04)\,\unit{\kelvin\per\minute}$}
In this section, the first dataset with heating rate
\begin{equation*}
    b_1=(1.69\pm0.04)\,\unit{\kelvin\per\minute}
\end{equation*}
will be analyzed. 
The temperature change over time is shown in \autoref{fig:temp_1}
\begin{figure}
   \centering
   \includegraphics[width =0.7\textwidth]{build/heating_rate_1.pdf}
   \caption{Temperature $T$ over time $t$.}
   \label{fig:temp_1}
\end{figure}
First the exponential background around the depolarization peak is fitted according to 
\begin{equation}
    I_{\text{background}}=a\exp(-b/T)+y_0
    \label{eq:background}
\end{equation}
in the Range $T\in[203.15\,\unit{\kelvin},240.45\,\unit{\kelvin}]\,\bigcup\,[274.15\,\unit{\kelvin},288.35\,\unit{\kelvin}]$ which yields
\begin{align*}
    a&=(4025\pm3373)\,\unit{\pico\ampere}.\\
    b&=(2123\pm214)\,\unit{\kelvin}\\
    y_0&=-(0.44\pm0.04)\,\unit{\pico\ampere}
\end{align*}
for the fit parameters. 
The measured depolarization current with background fit is depicted in \autoref{fig:Current_1}.
\begin{figure}
   \centering
   \includegraphics[width =0.7\textwidth]{build/Current1.pdf}
   \caption{Measured depolarization current and background fit at $b_1=(1.69\pm0.04)\,\unit{\kelvin\per\minute}$.}
   \label{fig:Current_1}
\end{figure}
This background is then removed from the measured current to obtain the depolarization current. 

\subsubsection{Polarization method}
The filtered current is then used to determine the activation energy $W$ according to the Polarizaiton method explained in \autoref{sec:theory}
by fitting 
\begin{equation*}
    \ln(I)=\frac{m}{T}+n
\end{equation*}
in the range $T\in[240.45\,\unit{\kelvin},T_{\text{max}}=259.25\,\unit{\kelvin}]$,
where $T_{\text{max}}$ is the Temperature corresponding to the maximum depolarization current. 
This fit is shown in \autoref{fig:Polarization_method1}
\begin{figure}
   \centering
   \includegraphics[width =0.7\textwidth]{build/Polarization_method1.pdf}
   \caption{Linearized depolarization current fitted according to the polarization method at $b_1=(1.69\pm0.04)\,\unit{\kelvin\per\minute}$.}
   \label{fig:Polarization_method1}
\end{figure}
This in turn yields
\begin{align*}
    m &= -(8114\pm 595)\,\unit{\kelvin}\\
    n &= 33\pm2
\end{align*}
therefore resulting in 
\begin{equation*}
W_{\text{Pol,}1}=-k_{\text{B}}m=(0.7\pm0.05)\,\unit{\electronvolt}
\end{equation*}
Using this result, the characteristic depolarization current $\tau_0$ can be calculated 
as
\begin{equation}
    \tau_0=\frac{k_{\text{B}}T_{\text{max}}^2}{W_{\text{Pol,}1}*b_1}\exp(-\frac{W_{\text{Pol,}1}}{k_{\text{B}}T_{\text{max}}})
    \label{eq:tau0}
\end{equation}
which results in 
\begin{equation*}
\tau_{0_{\text{Pol,}1}}=(1.3\pm3.0)\cdot10^{-13}\unit{\second}
\end{equation*}

\subsubsection{Integration method}
Analogously the activation energy $W$ is obtained using the integration method shown in \autoref{sec:theory}, by determining the integral
\begin{equation*}
F=\int^{T_{\text{max}}}_{T}{I(T)}%
\end{equation*}
and then fitting 
\begin{equation*}
\ln(\frac{F}{I})=\frac{m}{T}+n.
\end{equation*}
in the range $T\in[240.45\,\unit{\kelvin},T_{\text{max}}=259.25\,\unit{\kelvin}]$, which yields 
\begin{align*}
    m&=(17536\pm1040)\,\unit{\kelvin}\\
    n&=-67\pm4\\
\end{align*}
and results in
\begin{align*}
    W_{\text{Int,}1}&=(1.51\pm0.09)\,\unit{\electronvolt}  \\
    \tau_{0_{\text{Int,}1}}&=(1\pm4)\cdot10^{-29}\,\unit{\second} 
\end{align*}
where $tau_0$ was once again determined using \eqref{eq:tau0}. 
The corresponding fit is visualized in \autoref{fig:Integration_method1}.
\begin{figure}
   \centering
   \includegraphics[width =0.7\textwidth]{build/Integration_method1.pdf}
   \caption{Linearized depolarization current and fit according to integration method at $b_1=(1.69\pm0.04)\,\unit{\kelvin\per\minute}$.}
   \label{fig:Integration_method1}
\end{figure}





\subsection{Heating rate $b_2=(1.44\pm0.05)\,\unit{\kelvin\per\minute}$}
Now the dataset with heating rate
\begin{equation*}
    b_2=(1.44\pm0.05)\,\unit{\kelvin\per\minute}
\end{equation*}
will be analyzed. 
The temperature change over time is shown in \autoref{fig:temp_2}
\begin{figure}
   \centering
   \includegraphics[width =0.7\textwidth]{build/heating_rate_2.pdf}
   \caption{Temperature $T$ over time $t$.}
   \label{fig:temp_2}
\end{figure}
Again, the exponential background around the depolarization peak is fitted and removed first according to \eqref{eq:background}
in the Range $T\in[214.25\,\unit{\kelvin},242.55\,\unit{\kelvin}]\,\bigcup\,[271.15\,\unit{\kelvin},287.35\,\unit{\kelvin}]$ which yields
\begin{align*}
    a&=-(0.0002\pm0.0003)\,\unit{\pico\ampere}\\
    b&=-(-2201\pm338)\,\unit{\kelvin}\\
    y_0&=(1.7\pm0.3)\,\unit{\pico\ampere}
\end{align*}
for the fit parameters. 
The measured depolarization current with background fit is depicted in \autoref{fig:data_2}.
\begin{figure}
   \centering
   \includegraphics[width =0.7\textwidth]{build/Current2.pdf}
   \caption{Measured depolarization current $I$ and background fit at $b_2=(1.44\pm0.05)\,\unit{\kelvin\per\minute}$.}
   \label{fig:data_2}
\end{figure}

\subsubsection{Polarization method}
Analogously to the first dataset the activation energy $W$ is determined by fitting in the range $T\in[242.55\,\unit{\kelvin},T_{\text{max}}=258.45\,\unit{\kelvin}]$, 
with parameters 
\begin{align*}
    m&=-(6669\pm357)\,\unit{\kelvin}\\
    n&=27\pm1
\end{align*}
which resuls in 
\begin{align*}
W_{\text{Pol,}2}&=(0.575\pm0.03)\,\unit{\electronvolt} \\
\tau_{0_{\text{Pol,}2}}&=(4\pm6)\cdot10{-11}\,\unit{\second}
\end{align*}
The fit is shown in \autoref{fig:Polarization_method2}.
\begin{figure}
   \centering
   \includegraphics[width =0.7\textwidth]{build/Polarization_method2.pdf}
   \caption{Filtered depolarization current and fit according to polarization method at $b_2=(1.44\pm0.05)\,\unit{\kelvin\per\minute}$.}
   \label{fig:Polarization_method2}
\end{figure}

\subsubsection{Integration method}
Analogously the activation energy $W$ is obtained by integration and the following fit in the range $T\in[242.55\,\unit{\kelvin},T_{\text{max}}=258.45\,\unit{\kelvin}]$, which yields
\begin{align*}
    m&=(19458\pm1855)\,\unit{\kelvin}.\\
    n&=-76\pm7
\end{align*}
in the range $T\in$, which results in 
\begin{align*}
    W_{\text{Int,}2}&=(1.68\pm0.16)\,\unit{\electronvolt}  \\
    \tau_{0_{\text{Int,}2}}&=(0.5\pm3.5)\cdot10^{-35}\,\unit{\second}
\end{align*}
where $tau_0$ was once again determined using \eqref{eq:tau0}. 
The corresponding fit is visualized in \autoref{fig:Integration_method2}.
\begin{figure}
   \centering
   \includegraphics[width =0.7\textwidth]{build/Integration_method2.pdf}
   \caption{Filtered depolarization current and fit according to integration method at $b=  $.}
   \label{fig:Integration_method2}
\end{figure}

\subsection{Depolarization time}
From these results, which read
\begin{align*}
    W_{\text{Pol,}1}&=-k_{\text{B}}m=(0.7\pm0.05)\,\unit{\electronvolt}\\
    \tau_{0_{\text{Pol,}1}}&=(1.3\pm3.0)\cdot10^{-13}\unit{\second}\\
    W_{\text{Int,}1}&=(1.51\pm0.09)\,\unit{\electronvolt}  \\
    \tau_{0_{\text{Int,}1}}&=(1\pm4)\cdot10^{-29}\,\unit{\second}\\ 
    W_{\text{Pol,}2}&=(0.575\pm0.03)\,\unit{\electronvolt} \\
    \tau_{0_{\text{Pol,}2}}&=(4\pm6)\cdot10{-11}\,\unit{\second}\\
    W_{\text{Int,}2}&=(1.68\pm0.16)\,\unit{\electronvolt}  \\
    \tau_{0_{\text{Int,}2}}&=(0.5\pm3.5)\cdot10^{-35}
\end{align*}
the depolarization time $\tau(T)$ is plotted for all four cases and shown in \autoref{fig:Tau}. 
\begin{figure}
   \centering
   \includegraphics[width =0.7\textwidth]{build/tau.pdf}
   \caption{Resulting depolarization times $\tau(T)$.}
   \label{fig:Tau}
\end{figure}


% templates

%\begin{table}
%    \centering
%    \caption{.}
%    \label{tab:}
%    \sisetup{table-format = 3.0} 
%    \begin{tblr}{
%        colspec = {S S S},
%        row{1} = {guard, mode = math}
%    }
%        \toprule 
%        x \mathbin{/} \unit{\milli\meter} \\
%        \midrule
%        %data here
%        \bottomrule
%    \end{tblr}
%\end{table}

%\begin{figure}
%    \centering
%    \includegraphics[width =0.48\textwidth]{}
%    \caption{.}
%    \label{fig:}
%\end{figure}

%\begin{wrapfigure}[20]{r}{0.5\textwidth}
%    \begin{center}
%        \includegraphics[width =0.48\textwidth]{figures/}
%        \caption{}
%        \label{fig:}
%    \end{center}
%\end{wrapfigure}
\section{Analysis} %% Mention that sign of current is flipped
\label{sec:analysis}
In this chapter values for the heating rate $b$, activation Energy $W$, characteristic depolarization time scale $\tau_0$ 
and the temperature dependence of the depolarization time $\tau(T)$ will be calculated from two datasets with different heating rates
according to the polarization and integration method derived in \autoref{sec:theory}. 
The data is shown in the appendix. 
\subsection{Heating rate $b_1=(1.69\pm0.04)\,\unit{\kelvin\per\minute}$}
In this section, the first dataset with heating rate
\begin{equation*}
    b_1=(1.69\pm0.04)\,\unit{\kelvin\per\minute}
\end{equation*}
will be analyzed. 
The temperature change over time is shown in \autoref{fig:temp_1}
\begin{figure}
   \centering
   \includegraphics[width =0.7\textwidth]{build/heating_rate_1.pdf}
   \caption{Temperature $T$ over time $t$.}
   \label{fig:temp_1}
\end{figure}
First the exponential background around the depolarization peak is fitted according to 
\begin{equation}
    I_{\text{background}}=a\exp(-b/T)+y_0
    \label{eq:background}
\end{equation}
in the Range $T\in[203.15\,\unit{\kelvin},240.45\,\unit{\kelvin}]\,\bigcup\,[274.15\,\unit{\kelvin},288.35\,\unit{\kelvin}]$ which yields
\begin{align*}
    a&=(4025\pm3373)\,\unit{\pico\ampere}.\\
    b&=(2123\pm214)\,\unit{\kelvin}\\
    y_0&=-(0.44\pm0.04)\,\unit{\pico\ampere}
\end{align*}
for the fit parameters. 
The measured depolarization current with background fit is depicted in \autoref{fig:Current_1}.
\begin{figure}
   \centering
   \includegraphics[width =0.7\textwidth]{build/Current1.pdf}
   \caption{Measured depolarization current and background fit at $b_1=(1.69\pm0.04)\,\unit{\kelvin\per\minute}$.}
   \label{fig:Current_1}
\end{figure}
This background is then removed from the measured current to obtain the depolarization current. 
The corresponding values are listed in \autoref{tab:Current_1_no_background} and the resulting plot is shown in \autoref{fig:Current_1_no_background}. 


\begin{longtblr}[
    % \centering
    caption={Background-corrected depolarization current values at $b_1=(1.69\pm0.04)\,\unit{\kelvin\per\minute}$.},
    label={tab:Current_1_no_background},
    ]
    % \sisetup{table-format = 3.1}
    % \begin{tblr}%  
    {      
        colspec = {S[table-format = 3.1] Q[c, mode=math]},
        row{1} = {guard, mode = math},
        }
        \toprule
        T \mathbin{/} \unit{\kelvin} & I \mathbin{/} \unit{\pico\ampere} \\
        \midrule
        203.1 &  0.12\pm0.16  \\
        202.5 &  0.14\pm0.16  \\
        202.5 &  0.13\pm0.16  \\
        202.7 &  0.11\pm0.16  \\
        203.0 &  0.08\pm0.16  \\
        203.3 &  0.06\pm0.16  \\
        203.7 &  0.03\pm0.17  \\
        204.0 &  0.03\pm0.17  \\
        204.5 &  0.01\pm0.17  \\
        205.1 &  0.01\pm0.18  \\
        205.5 &  -0.00\pm0.18  \\
        206.1 &  -0.02\pm0.19  \\
        206.7  &  -0.02\pm0.19  \\
        207.5 &  -0.03\pm0.20  \\
        208.2  &  -0.04\pm0.20  \\
        209.0 &  -0.05\pm0.21  \\
        210.0 &  -0.06\pm0.22  \\
        211.2  &  -0.07\pm0.23  \\
        212.5 &  -0.08\pm0.25  \\
        214.1 &  -0.09\pm0.26  \\
        216.0 &  -0.10\pm0.28  \\
        218.0 &  -0.10\pm0.31  \\
        220.45 &  -0.12\pm0.34  \\
        222.5 &  -0.1\pm0.4  \\
        224.6 &  -0.2\pm0.4  \\
        226.7  &  -0.0\pm0.4  \\
        228.7  &  -0.1\pm0.5  \\
        230.8  &  -0.1\pm0.5  \\
        232.8  &  -0.1\pm0.6  \\
        234.7  &  -0.0\pm0.6  \\
        236.5 &  0.0\pm0.6  \\
        238.5 &  0.1\pm0.7  \\
        240.5 &  0.2\pm0.7  \\
        242.2  &  0.4\pm0.8  \\
        244.0 &  0.6\pm0.8  \\
        245.8  &  0.9\pm0.9  \\
        247.6 &  1.2\pm0.9  \\
        249.3  &  1.6\pm1.0  \\
        251.0 &  2.2\pm1.0  \\
        252.7  &  2.6\pm1.1  \\
        254.3  &  3.2\pm1.1  \\
        256.0   &  3.7\pm1.2  \\
        257.7 &  4.1\pm1.3  \\
        259.3 &  4.2\pm1.3  \\
        260.8  &  4.1\pm1.4  \\
        262.3  &  3.6\pm1.4  \\
        264.0   &  2.9\pm1.5  \\
        265.5 &  2.3\pm1.6  \\
        267.0 &  1.4\pm1.6  \\
        268.5 &  0.8\pm1.7  \\
        267.0 &  0.3\pm1.8  \\
        271.5 &  0.0\pm1.9  \\
        272.8  &  -0.1\pm1.9  \\
        274.2 &  -0.3\pm2.0  \\
        275.5   &  -0.3\pm2.1  \\
        277.0   &  -0.4\pm2.2  \\
        278.5 &  -0.3\pm2.2  \\
        279.8  &  -0.3\pm2.3  \\
        281.3  &  -0.3\pm2.4  \\
        282.7 &  -0.2\pm2.5  \\
        284.0  &  -0.2\pm2.6  \\
        285.3  &  -0.2\pm2.7  \\
        286.8  &  -0.3\pm2.8  \\
        288.3  &  -0.3\pm2.9  \\
        \bottomrule
    % \end{tblr}
\end{longtblr}

\begin{figure}
   \centering
   \includegraphics[width =0.7\textwidth]{build/Current1_no background.pdf}
   \caption{Background corrected depolarization current at $b_1=(1.69\pm0.04)\,\unit{\kelvin\per\minute}$.}
   \label{fig:Current_1_no_background}
\end{figure}

\subsubsection{Polarization method}
The filtered current is then used to determine the activation energy $W$ according to the Polarizaiton method explained in \autoref{sec:theory}
by fitting 
\begin{equation*}
    \ln(I/1\,\unit{\pico\ampere})=\frac{m}{T}+n
\end{equation*}
in the range $T\in[240.45\,\unit{\kelvin},T_{\text{max}}=259.25\,\unit{\kelvin}]$,
where $T_{\text{max}}$ is the Temperature corresponding to the maximum depolarization current. 
The corresponding values are shown in \autoref{tab:Polarization_method_1}.
\begin{longtblr}[
        % \centering
        caption={Values used for the polarization method at $b_1=(1.69\pm0.04)\,\unit{\kelvin\per\minute}$.},
        label={tab:Polarization_method_1},
        ]
        % \sisetup{table-format = 3.1}
        % \begin{tblr}%  
        {      
            colspec = {S[table-format = 3.1] S[table-format = 3.1] Q[c, mode=math]},
            row{1} = {guard, mode = math},
            }
            \toprule
            T \mathbin{/} \unit{\kelvin} & 1/T \cdot 10^{-3} \mathbin{/} \unit{\kelvin^{-1}} & \ln(I/1\,\unit{\pico\ampere})\\
            \midrule
            240.4 & 4.159 & -1.6 \pm 3.6 \\
            242.2 & 4.128 & -1.0 \pm 2.1 \\
            244.0 & 4.098 & -0.6 \pm 1.4 \\
            245.8 & 4.068 & -0.1 \pm 1.0 \\
            247.6 & 4.038 & 0.2 \pm 0.8 \\
            249.3 & 4.010 & 0.5 \pm 0.6 \\
            251.0 & 3.983 & 0.8 \pm 0.5 \\
            252.7 & 3.956 & 1.0 \pm 0.4 \\
            254.3 & 3.932 & 1.2 \pm 0.4 \\
            256.0 & 3.905 & 1.3 \pm 0.3 \\
            257.6 & 3.881 & 1.4 \pm 0.3 \\
            \bottomrule
        % \end{tblr}
\end{longtblr}
This fit is shown in \autoref{fig:Polarization_method1}.
\begin{figure}
   \centering
   \includegraphics[width =0.7\textwidth]{build/Polarization_method1.pdf}
   \caption{Linearized depolarization current fitted according to the polarization method at $b_1=(1.69\pm0.04)\,\unit{\kelvin\per\minute}$.}
   \label{fig:Polarization_method1}
\end{figure}


This in turn yields
\begin{align*}
    m &= -(8114\pm 595)\,\unit{\kelvin}\\
    n &= 33\pm2 
\end{align*}
therefore resulting in 
\begin{equation*}
W_{\text{Pol,}1}=-k_{\text{B}}m=(0.7\pm0.05)\,\unit{\electronvolt}
\end{equation*}
Using this result, the characteristic depolarization current $\tau_0$ can be calculated 
as
\begin{equation}
    \tau_0=\frac{k_{\text{B}}T_{\text{max}}^2}{W_{\text{Pol,}1}*b_1}\exp(-\frac{W_{\text{Pol,}1}}{k_{\text{B}}T_{\text{max}}})
    \label{eq:tau0}
\end{equation}
which results in 
\begin{equation*}
\tau_{0_{\text{Pol,}1}}=(1.3\pm3.0)\cdot10^{-13}\unit{\second}
\end{equation*}
where $T_{\text{max}}=259.25\,\unit{\kelvin}$ and $b_1=(1.69\pm0.04)\,\unit{\kelvin\per\minute}$ were used as introduced before. 

\subsubsection{Integration method}
Analogously the activation energy $W$ is obtained using the integration method shown in \autoref{sec:theory}, by determining the integral
\begin{equation*}
F=\int^{T_{\text{max}}}_{T}{I(T)}%
\end{equation*}
and then fitting 
\begin{equation*}
\ln(\frac{F}{I})=\frac{m}{T}+n.
\end{equation*}
in the range $T\in[240.45\,\unit{\kelvin},T_{\text{max}}=259.25\,\unit{\kelvin}]$, which yields 
\begin{align*}
    m&=(17536\pm1040)\,\unit{\kelvin}\\
    n&=-67\pm4\\
\end{align*}
and results in
\begin{align*}
    W_{\text{Int,}1}&=(1.51\pm0.09)\,\unit{\electronvolt}  \\
    \tau_{0_{\text{Int,}1}}&=(1\pm4)\cdot10^{-29}\,\unit{\second} 
\end{align*}
where $\tau_0$ was once again determined using \eqref{eq:tau0} using $T_{\text{max}}=259.25\,\unit{\kelvin}$ and $b_1=(1.69\pm0.04)\,\unit{\kelvin\per\minute}$. 
The corresponding fit is visualized in \autoref{fig:Integration_method1}.
\begin{figure}
   \centering
   \includegraphics[width =0.7\textwidth]{build/Integration_method1.pdf}
   \caption{Linearized depolarization current and fit according to integration method at $b_1=(1.69\pm0.04)\,\unit{\kelvin\per\minute}$.}
   \label{fig:Integration_method1}
\end{figure}

%%% CORR: Table with all values
The values used for the calculations are shown in \autoref{tab:Integration_method_1}. 
\begin{longtblr}[
        % \centering
        caption={Values used for the integration method at $b_1=(1.69\pm0.04)\,\unit{\kelvin\per\minute}$.},
        label={tab:Integration_method_1},
        ]
        % \sisetup{table-format = 3.1}
        % \begin{tblr}%  
        {      
            colspec = {S[table-format = 3.1] S[table-format = 3.1] Q[c, mode=math] S[table-format = 2.1] Q[c, mode=math]},
            row{1} = {guard, mode = math},
            }
            \toprule
            T \mathbin{/} \unit{\kelvin} & 1/T \cdot 10^{-3} \mathbin{/} \unit{\kelvin^{-1}} & I \mathbin{/} \unit{\pico\ampere} & F \mathbin{/} \unit{\pico\ampere\second} & \ln(F/I\cdot1\,\unit{\per\second})\\
            \midrule
            240.4 & 4.159 & -1.6 \pm 3.6 & 37.6 & 5.2 \pm 3.6 \\
            242.2 & 4.128 & -1.0 \pm 2.1 & 37.1 & 4.6 \pm 2.1 \\
            244.0 & 4.098 & -0.6 \pm 1.4 & 36.3 & 4.2 \pm 1.4 \\
            245.8 & 4.068 & -0.1 \pm 1.0 & 35.0 & 3.7 \pm 1.0 \\
            247.6 & 4.038 & 0.2 \pm 0.8 & 33.1 & 3.3 \pm 0.8 \\
            249.3 & 4.010 & 0.5 \pm 0.6 & 30.7 & 2.9 \pm 0.6 \\
            251.0 & 3.983 & 0.8 \pm 0.5 & 27.5 & 2.5 \pm 0.5 \\
            252.7 & 3.956 & 1.0 \pm 0.4 & 23.4 & 2.2 \pm 0.4 \\
            254.3 & 3.932 & 1.2 \pm 0.4 & 18.8 & 1.8 \pm 0.4 \\
            256.0 & 3.905 & 1.3 \pm 0.3 & 12.9 & 1.2 \pm 0.3 \\
            257.6 & 3.881 & 1.4 \pm 0.3 & 6.6 & 0.5 \pm 0.3 \\
            \bottomrule
        % \end{tblr}
\end{longtblr}


\subsection{Heating rate $b_2=(1.44\pm0.05)\,\unit{\kelvin\per\minute}$}
Now the dataset with heating rate
\begin{equation*}
    b_2=(1.44\pm0.05)\,\unit{\kelvin\per\minute}
\end{equation*}
will be analyzed. 
The temperature change over time is shown in \autoref{fig:temp_2}
\begin{figure}
   \centering
   \includegraphics[width =0.7\textwidth]{build/heating_rate_2.pdf}
   \caption{Temperature $T$ over time $t$.}
   \label{fig:temp_2}
\end{figure}
Again, the exponential background around the depolarization peak is fitted and removed first according to \eqref{eq:background}
in the Range $T\in[214.25\,\unit{\kelvin},242.55\,\unit{\kelvin}]\,\bigcup\,[271.15\,\unit{\kelvin},287.35\,\unit{\kelvin}]$ which yields
\begin{align*}
    a&=-(0.0002\pm0.0003)\,\unit{\pico\ampere}\\
    b&=-(-2201\pm338)\,\unit{\kelvin}\\
    y_0&=(1.7\pm0.3)\,\unit{\pico\ampere}
\end{align*}
for the fit parameters. 
The measured depolarization current with background fit is depicted in \autoref{fig:data_2}.
\begin{figure}
   \centering
   \includegraphics[width =0.7\textwidth]{build/Current2.pdf}
   \caption{Measured depolarization current $I$ and background fit at $b_2=(1.44\pm0.05)\,\unit{\kelvin\per\minute}$.}
   \label{fig:data_2}
\end{figure}
The corrected depolarization current is listed in \autoref{tab:Current_2_no_background} 
and the resulting plot is shown in \autoref{fig:Current_2_no_background}. 
\begin{longtblr}[
    % \centering
    caption={Background-corrected depolarization current values at $b_2=(1.44\pm0.05)\,\unit{\kelvin\per\minute}$.},
    label={tab:Current_2_no_background},
    ]
    % \sisetup{table-format = 3.1}
    % \begin{tblr}%  
    {      
        colspec = {S[table-format = 3.1] Q[c, mode=math]},
        row{1} = {guard, mode = math},
        }
        \toprule
        T \mathbin{/} \unit{\kelvin} & I \mathbin{/} \unit{\pico\ampere} \\
        \midrule
            214.2 &  0.8 \pm 13.6   \\
            216.4 &  0.3 \pm 12.2   \\
            218.4 &  -0.2\pm 11.0   \\
            220.5 &  -0.7\pm 10.0   \\
            222.5 &  -0.8\pm 9.1   \\
            224.5 &  -1.0\pm 8.3   \\
            226.6 &  0.5 \pm 7.5   \\
            228.3 &  -0.2\pm 7.0   \\
            230.2 &  -0.3\pm 6.4   \\
            232.1 &  -0.2\pm 5.9   \\
            233.9 &  0.1 \pm 5.5   \\
            235.7 &  0.7 \pm 5.1   \\
            237.5 &  0.6 \pm 4.7   \\
            239.1 &  0.6 \pm 4.4   \\
            240.8 &  0.6 \pm 4.1   \\
            242.5 &  0.7 \pm 3.9   \\
            244.0 &  0.9 \pm 3.7   \\
            245.6 &  1.0 \pm 3.4   \\
            247.2 &  1.2 \pm 3.2   \\
            248.7 &  1.5 \pm 3.1   \\
            250.2 &  1.8 \pm 2.9   \\
            251.6 &  2.2 \pm 2.8   \\
            253.0 &  2.6 \pm 2.6   \\
            254.5 &  2.9 \pm 2.5   \\
            255.9 &  3.2 \pm 2.4   \\
            257.1 &  3.4 \pm 2.3   \\
            258.4 &  3.4 \pm 2.2   \\
            259.9 &  3.1 \pm 2.1   \\
            261.0 &  2.8 \pm 2.0   \\
            262.3 &  2.3 \pm 1.9   \\
            263.6 &  1.8 \pm 1.8   \\
            264.9 &  1.3 \pm 1.8   \\
            266.1 &  0.7 \pm 1.7   \\
            267.4 &  0.4 \pm 1.6   \\
            268.6 &  0.1 \pm 1.6   \\
            269.9 &  -0.2\pm 1.5   \\
            271.1 &  -0.2\pm 1.4   \\
            272.4 &  -0.3\pm 1.4   \\
            273.6 &  -0.2\pm 1.3   \\
            274.8 &  -0.2\pm 1.3   \\
            275.9 &  -0.2\pm 1.3   \\
            277.0 &  -0.1\pm 1.2   \\
            278.1 &  -0.1\pm 1.2   \\
            279.2 &  -0.1\pm 1.1   \\
            280.2 &  -0.0\pm 1.1   \\
            281.3 &  0.0 \pm 1.1   \\
            282.2 &  0.0 \pm 1.1   \\
            283.3 &  0.1 \pm 1.0   \\
            284.3 &  0.1 \pm 1.0   \\
            285.4 &  0.1 \pm 1.0   \\
            286.4 &  0.1 \pm 0.9   \\
            287.3 &  0.2 \pm 0.9   \\
        \bottomrule
    % \end{tblr}
\end{longtblr}
\begin{figure}
   \centering
   \includegraphics[width =0.7\textwidth]{build/Current2_no background.pdf}
   \caption{Background corrected depolarization current at $b_2=(1.44\pm0.05)\,\unit{\kelvin\per\minute}$.}
   \label{fig:Current_2_no_background}
\end{figure}

\subsubsection{Polarization method}
Analogously to the first dataset the activation energy $W$ is determined by fitting in the range $T\in[242.55\,\unit{\kelvin},T_{\text{max}}=258.45\,\unit{\kelvin}]$, 
with parameters 
\begin{align*}
    m&=-(6669\pm357)\,\unit{\kelvin}\\
    n&=27\pm1
\end{align*}
which resuls in 
\begin{align*}
W_{\text{Pol,}2}&=(0.575\pm0.03)\,\unit{\electronvolt} \\
\tau_{0_{\text{Pol,}2}}&=(4\pm6)\cdot10{-11}\,\unit{\second}
\end{align*}
where $T_{\text{max}}=258.45\,\unit{\kelvin}$ and $b_2=(1.44\pm0.05)\,\unit{\kelvin\per\minute}$ were used. 
The corresponding values are shown in \autoref{tab:Polarization_method_2}.
\begin{longtblr}[
        % \centering
        caption={Values used for the polarization method at $b_2=(1.44\pm0.05)\,\unit{\kelvin\per\minute}$.},
        label={tab:Polarization_method_2},
]
        % \sisetup{table-format = 3.1}
        % \begin{tblr}%  
        {      
            colspec = {S[table-format = 3.1] S[table-format = 3.1] Q[c, mode=math]},
            row{1} = {guard, mode = math},
            }
            \toprule
            T \mathbin{/} \unit{\kelvin} & 1/T \cdot 10^{-3} \mathbin{/} \unit{\kelvin^{-1}} & \ln(I/1\,\unit{\pico\ampere})\\
            \midrule
            242.5 & 4.123 & -0.4 \pm 5.5 \\
            244.0 & 4.098 & -0.1 \pm 4.2 \\
            245.6 & 4.071 & -0.0 \pm 3.4 \\
            247.2 & 4.044 & 0.2 \pm 2.7 \\
            248.7 & 4.020 & 0.4 \pm 2.0 \\
            250.2 & 3.996 & 0.6 \pm 1.6 \\
            251.6 & 3.974 & 0.8 \pm 1.2 \\
            253.0 & 3.952 & 0.9 \pm 1.0 \\
            254.5 & 3.929 & 1.1 \pm 0.9 \\
            255.9 & 3.907 & 1.1 \pm 0.8 \\
            \bottomrule
        % \end{tblr}
\end{longtblr}
The fit is shown in \autoref{fig:Polarization_method2}.
\begin{figure}
   \centering
   \includegraphics[width =0.7\textwidth]{build/Polarization_method2.pdf}
   \caption{Filtered depolarization current and fit according to polarization method at $b_2=(1.44\pm0.05)\,\unit{\kelvin\per\minute}$.}
   \label{fig:Polarization_method2}
\end{figure}

\subsubsection{Integration method}
Analogously the activation energy $W$ is obtained by integration and the following fit in the range $T\in[242.55\,\unit{\kelvin},T_{\text{max}}=258.45\,\unit{\kelvin}]$, which yields
\begin{align*}
    m&=(19458\pm1855)\,\unit{\kelvin}.\\
    n&=-76\pm7
\end{align*}
in the range $T\in$, which results in 
\begin{align*}
    W_{\text{Int,}2}&=(1.68\pm0.16)\,\unit{\electronvolt}  \\
    \tau_{0_{\text{Int,}2}}&=(0.5\pm3.5)\cdot10^{-35}\,\unit{\second}
\end{align*}
where $\tau_0$ was once again determined using \eqref{eq:tau0} and $T_{\text{max}}=258.45\,\unit{\kelvin}$ and $b_2=(1.44\pm0.05)\,\unit{\kelvin\per\minute}$. 
The corresponding fit is visualized in \autoref{fig:Integration_method2}.
\begin{figure}
   \centering
   \includegraphics[width =0.7\textwidth]{build/Integration_method2.pdf}
   \caption{Filtered depolarization current and fit according to integration method at $b_2=(1.44\pm0.05)\,\unit{\kelvin\per\minute}$.}
   \label{fig:Integration_method2}
\end{figure}

The values used for the calculations are shown in \autoref{tab:Integration_method_2}. 
\begin{longtblr}[
        % \centering
        caption={Values used for the integration method at $b_2=(1.44\pm0.05)\,\unit{\kelvin\per\minute}$.},
        label={tab:Integration_method_2},
        ]
        % \sisetup{table-format = 3.1}
        % \begin{tblr}%  
        {      
            colspec = {S[table-format = 3.1] S[table-format = 3.1] Q[c, mode=math] S[table-format = 2.1] Q[c, mode=math]},
            row{1} = {guard, mode = math},
            }
            \toprule
            T \mathbin{/} \unit{\kelvin} & 1/T \cdot 10^{-3} \mathbin{/} \unit{\kelvin^{-1}} & I \mathbin{/} \unit{\pico\ampere} & F \mathbin{/} \unit{\pico\ampere\second} & \ln(F/I\cdot1\,\unit{\per\second})\\
            \midrule
            242.5 & 4.123e-03 & -0.4 \pm 5.5 & 27.5 & 3.7 \pm 5.5 \\
            244.0 & 4.098e-03 & -0.1 \pm 4.2 & 26.3 & 3.4 \pm 4.2 \\
            245.6 & 4.071e-03 & -0.0 \pm 3.4 & 24.8 & 3.2 \pm 3.4 \\
            247.2 & 4.044e-03 & 0.2 \pm 2.7 & 23.0 & 2.9 \pm 2.7 \\
            248.7 & 4.020e-03 & 0.4 \pm 2.0 & 21.0 & 2.6 \pm 2.0 \\
            250.2 & 3.996e-03 & 0.6 \pm 1.6 & 18.5 & 2.3 \pm 1.6 \\
            251.6 & 3.974e-03 & 0.8 \pm 1.2 & 15.7 & 1.9 \pm 1.2 \\
            253.0 & 3.952e-03 & 0.9 \pm 1.0 & 12.3 & 1.6 \pm 1.0 \\
            254.5 & 3.929e-03 & 1.1 \pm 0.9 & 8.2 & 1.0 \pm 0.9 \\
            255.9 & 3.907e-03 & 1.1 \pm 0.8 & 3.9 & 0.2 \pm 0.8 \\
            \bottomrule
        % \end{tblr}
\end{longtblr}

\subsection{Depolarization time}
From these results, which read
\begin{align*}
    W_{\text{Pol,}1}&=-k_{\text{B}}m=(0.7\pm0.05)\,\unit{\electronvolt}\\
    \tau_{0_{\text{Pol,}1}}&=(1.3\pm3.0)\cdot10^{-13}\unit{\second}\\
    W_{\text{Int,}1}&=(1.51\pm0.09)\,\unit{\electronvolt}  \\
    \tau_{0_{\text{Int,}1}}&=(1\pm4)\cdot10^{-29}\,\unit{\second}\\ 
    W_{\text{Pol,}2}&=(0.575\pm0.03)\,\unit{\electronvolt} \\
    \tau_{0_{\text{Pol,}2}}&=(4\pm6)\cdot10^{-11}\,\unit{\second}\\
    W_{\text{Int,}2}&=(1.68\pm0.16)\,\unit{\electronvolt}  \\
    \tau_{0_{\text{Int,}2}}&=(0.5\pm3.5)\cdot10^{-35}\,\unit{\second}
\end{align*}
the depolarization time $\tau(T)$ is plotted for all four cases and shown in \autoref{fig:Tau}. 
\begin{figure}
   \centering
   \includegraphics[width =0.7\textwidth]{build/tau.pdf}
   \caption{Resulting depolarization times $\tau(T)$.}
   \label{fig:Tau}
\end{figure}


% templates

%\begin{table}
%    \centering
%    \caption{.}
%    \label{tab:}
%    \sisetup{table-format = 3.0} 
%    \begin{tblr}{
%        colspec = {S S S},
%        row{1} = {guard, mode = math}
%    }
%        \toprule 
%        x \mathbin{/} \unit{\milli\meter} \\
%        \midrule
%        %data here
%        \bottomrule
%    \end{tblr}
%\end{table}

%\begin{figure}
%    \centering
%    \includegraphics[width =0.48\textwidth]{}
%    \caption{.}
%    \label{fig:}
%\end{figure}

%\begin{wrapfigure}[20]{r}{0.5\textwidth}
%    \begin{center}
%        \includegraphics[width =0.48\textwidth]{figures/}
%        \caption{}
%        \label{fig:}
%    \end{center}
%\end{wrapfigure}
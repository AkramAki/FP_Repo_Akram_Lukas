\section{Analysis} %% Mention that sign of current is flipped
\label{sec:analysis}
In this chapter values for the heating rate $b$, activation Energy $W$, characteristic depolarization time scale $\tau_0$ 
and the temperature dependence of the depolarization time $\tau(T)$ will be calculated from two datasets with different heating rates
according to the polarization and integration method derived in CHAPTER. %% Reference and check tau 0 name
The data is shown in the APPENDIX. %% Reference
\subsection{Heating rate $b=  \,\unit{\kelvin\per\minute}$}
In this section, the first dataset with heating rate
\begin{equation*}
    b=
\end{equation*}
will be analyzed. 
First the exponential background around the depolarization peak is fitted according to 
\begin{equation*}
    I_{\text{background}}=
\end{equation*}
in the Range $T\in    $ which yields
\begin{align*}
    .
\end{align*}
for the fit parameters. 
The measured depolarization current with background fit is depicted in \autoref{fig:data_1}.
%\begin{figure}
%    \centering
%    \includegraphics[width =0.48\textwidth]{build/Current1.pdf}
%    \caption{Measured depolarization current and background fit at $b=  $.}
%    \label{fig:data_1}
%\end{figure}
This background is then removed from the measured current to obtain the depolarization current. 

\subsubsection{Polarization method}
The filtered current is then used to determine the activation energy $W$ from EQUATION %% Reference
by fitting 
\begin{equation*}
.
\end{equation*}
in the range $T\in   $,
where $T_{\text{max}}$ is the Temperature corresponding to the maximum depolarization current. 
This fit is shown in \autoref{fig:Polarization_method1}
%\begin{figure}
%    \centering
%    \includegraphics[width =0.48\textwidth]{build/Polarization_method1.pdf}
%    \caption{Filtered depolarization current and fit according to polarization method at $b=  $.}
%    \label{fig:Polarization_method1}
%\end{figure}
This in turn yields
\begin{align*}
.
\end{align*}
therefore resulting in 
\begin{equation*}
W_{\text{Pol,}1}=
\end{equation*}
Using this result, the characteristic depolarization current $\tau_0$ can be calculated when using EQUATION %% Reference
which results in 
\begin{equation*}
\tau_{0_{\text{Pol,}1}}=
\end{equation*}

\subsubsection{Integration method}
Analogously the activation energy $W$ is obtained using the integration method shown in EQUATION, by determining the integral
\begin{equation*}
\int{I(T)}^{}_{} =  %
\end{equation*}
and then fitting 
\begin{equation*}
.
\end{equation*}
in the range $T\in$, which yields 
\begin{align*}
.
\end{align*}
and results in
\begin{align*}
    W_{\text{Int,}1}=  \\
    \tau_{0_{\text{Int,}1}}= 
\end{align*}
where $tau_0$ was once again determined using EQUATION. %% Reference
The corresponding fit is visualized in \autoref{fig:Integration_method1}.
%\begin{figure}
%    \centering
%    \includegraphics[width =0.48\textwidth]{build/Integration_method1.pdf}
%    \caption{Filtered depolarization current and fit according to integration method at $b=  $.}
%    \label{fig:Integration_method1}
%\end{figure}





\subsection{Heating rate $b=  \,\unit{\kelvin\per\minute}$}
Now the dataset with heating rate
\begin{equation*}
    b=
\end{equation*}
will be analyzed. 
Again, the exponential background around the depolarization peak is fitted and removed first according to 
\begin{equation*}
    I_{\text{background}}=
\end{equation*}
in the Range $T\in    $ which yields
\begin{align*}
.
\end{align*}
for the fit parameters. 
The measured depolarization current with background fit is depicted in \autoref{fig:data_2}.
%\begin{figure}
%    \centering
%    \includegraphics[width =0.48\textwidth]{build/Current2.pdf}
%    \caption{Measured depolarization current and background fit at $b=  $.}
%    \label{fig:data_2}
%\end{figure}

\subsubsection{Polarization method}
Analogously to the first dataset the activation energy $W$ is determined by fitting in the range $T\in   $, 
with parameters 
\begin{align*}
.
\end{align*}
which resuls in 
\begin{align*}
W_{\text{Pol,}2}= \\
\tau_{0_{\text{Pol,}2}}=
\end{align*}
The fit is shown in \autoref{fig:Polarization_method2}.
%\begin{figure}
%    \centering
%    \includegraphics[width =0.48\textwidth]{build/Polarization_method2.pdf}
%    \caption{Filtered depolarization current and fit according to polarization method at $b=  $.}
%    \label{fig:Polarization_method2}
%\end{figure}

\subsubsection{Integration method}
Analogously the activation energy $W$ is obtained by integration and the following fit in the range $T\in$, which yields
\begin{align*}
.
\end{align*}
in the range $T\in$, which results in 
\begin{align*}
    W_{\text{Int,}2}=  \\
    \tau_{0_{\text{Int,}2}}= 
\end{align*}
where $tau_0$ was once again determined using EQUATION. %% Reference
The corresponding fit is visualized in \autoref{fig:Integration_method2}.
%\begin{figure}
%    \centering
%    \includegraphics[width =0.48\textwidth]{build/Integration_method2.pdf}
%    \caption{Filtered depolarization current and fit according to integration method at $b=  $.}
%    \label{fig:Integration_method2}
%\end{figure}

\subsection{Depolarization time}
From these results, summarized in TABLE %% Reference
the depolarization time $\tau(T)$ is plotted for all four cases and shown in \autoref{fig:Tau}. 
The corresponding fit is visualized in \autoref{fig:Integration_method1}.
%\begin{figure}
%    \centering
%    \includegraphics[width =0.48\textwidth]{build/tau.pdf}
%    \caption{Resulting depolarization times $\tau(T)$.}
%    \label{fig:Tau}
%\end{figure}


% templates

%\begin{table}
%    \centering
%    \caption{.}
%    \label{tab:}
%    \sisetup{table-format = 3.0} 
%    \begin{tblr}{
%        colspec = {S S S},
%        row{1} = {guard, mode = math}
%    }
%        \toprule 
%        x \mathbin{/} \unit{\milli\meter} \\
%        \midrule
%        %data here
%        \bottomrule
%    \end{tblr}
%\end{table}

%\begin{figure}
%    \centering
%    \includegraphics[width =0.48\textwidth]{}
%    \caption{.}
%    \label{fig:}
%\end{figure}

%\begin{wrapfigure}[20]{r}{0.5\textwidth}
%    \begin{center}
%        \includegraphics[width =0.48\textwidth]{figures/}
%        \caption{}
%        \label{fig:}
%    \end{center}
%\end{wrapfigure}
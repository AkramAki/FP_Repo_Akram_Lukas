\section{Discussion}
\label{sec:discussion}

The experiment successfully demonstrates the thermally stimulated relaxation of
defect-dipoles in a doped ionic crystal. For both heating rates, a well-defined
depolarisation peak is observed (see \autoref{fig:Current_1} and \autoref{fig:data_2}),
indicating that the preparation procedure and the alignment of dipoles were largely
successful. Nevertheless, the quantitative results show several systematic deviations that
must be addressed.

A first point of discussion concerns the extracted activation energies. For both datasets,
Method~I (low-temperature approximation) yields significantly smaller activation energies
than Method~II (integral method). For the first heating rate, the values differ by more
than a factor of two, and similar discrepancies appear for the second dataset. Such a
mismatch suggests that either the assumptions of Method~I are not fully satisfied in the
chosen fitting region, or that Method~II is strongly affected by noise.

The low-temperature approximation requires a temperature interval in which the current follows a simple exponential increase with temperature. However, the semi-logarithmic fits in
\autoref{fig:Polarization_method1} and \autoref{fig:Polarization_method2} show noticeable curvature,
indicating that the exponential regime is rather narrow. This makes the fitted slopes
sensitive to small variations in the background-corrected current.

Method~II, on the other hand, amplifies fluctuations because the integral
\[
    S(T) = \int_T^{T_{\text{max}}} I(T')\,\text{d}T'
\]
becomes small near $T_{\text{max}}$. As a result, the ratio $S(T)/I(T)$ is very
susceptible to noise in both the integral and the instantaneous current. This is clearly
visible in the scattered data points in \autoref{fig:Integration_method1} and
\autoref{fig:Integration_method2}. Consequently, Method~II produces activation energies and
especially relaxation times $\tau_0$ spanning unphysical ranges, including values as small
as $10^{-35}\,\text{s}$.

A second systematic uncertainty arises from the background subtraction. Although the
exponential fits in \autoref{fig:Current_1} and \autoref{fig:data_2} describe the raw data reasonably
well, small deviations from the true background propagate directly into both evaluation
methods. Because the methods rely on logarithmic transformations, even minor
overestimation or underestimation of the background can cause large shifts in the fitted
slopes.

The characteristic relaxation times $\tau_0$, extracted using the extremum condition
\eqref{eq:tau0}, also show significant spread. While Method~I yields values that are at
least of the correct order of magnitude ($10^{-13}$-$10^{-11}\,\text{s}$), Method~II
produces values that are many orders of magnitude too small. This strong sensitivity is
expected, as $\tau_0$ depends exponentially on $W$ and inversely on the heating rate, so
even a small deviation in $W$ leads to large multiplicative errors in $\tau_0$.

Finally, a comparison of the two heating rates reveals limited reproducibility. Both the
shape of the depolarisation peak and the extracted parameters differ for the two datasets.
Possible reasons include slight variations in thermal contact between the sample and
coldfinger, differences in the effective polarisation time, or temperature gradients in the
vacuum chamber, consistent with the experimental layout shown in the setup
(see page~4 of the report). 

In summary, the experiment qualitatively confirms the expected behaviour of thermally
stimulated depolarisation, but the quantitative agreement between evaluation methods is
limited. The main sources of uncertainty are noise amplification in the integral method,
the narrowness of the low-temperature exponential regime, and imperfections in
background subtraction. Improved temperature control, higher-resolution current
measurements, and more carefully selected fitting intervals would likely reduce these
systematic deviations in future measurements.

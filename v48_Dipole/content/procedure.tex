\section{Experimental Setup and Procedure}
\label{sec:procedure}
In this chapter the setup of the experiment and the measuring procedure will be described. 

\subsection{Setup}
The experiment is conducted with a Potassium Bromide sample, 
which is placed inside of a vacuum chamber that is cooled using a dewar flask filled with liquid nitrogen. 
Thermal contact is ensured by a coldfinger. 
Inside the chamber a plate is placed on top of the sample and supplied with a voltage, to create a capacitor together with the recipients floor. 
Furthermore there is a heating wire inside of the recipient, which can be used to heat the sample. 
Multiple devices are connected to measure pressure, temperature, relaxation current or supply current for the heating or power the capacitor. 
\autoref{fig:setup} shows the setup. 
\begin{figure}
   \centering
   \includegraphics[width =0.48\textwidth]{content/images/setup.png}
   \caption{Components of the setup \cite{v48}.}
   \label{fig:setup}
\end{figure}

\subsection{Procedure}
First, the sample is heated to $50\,\unit{\celsius}$ and the capacitor is charged with $U_{\text{c}}=950\,\unit{\volt}$, to align the dipoles. 
After waiting for $15$ minutes, which is done to ensure most of the dipoles are aligned, liquid nitrogen is filled into the 
dewar flask and cooling starts. 
When the temperature reaches around $-50\,\unit{celsius}$ the capacitor is turned back of and short circuited for $5$ minutes. 
Then a picoamperemeter is connected to the recipient and the heating is turned back on as soon as the amperemeter output is constant.  
While the sample heats up to $20\,\unit{\kelvin}$ the temperature and relaxation current are measured every minute. 
To reach a constant heating rate the heating current might be adjusted during measurement. \\
This procedure is repeated with two different heating rates. 